\section{Introduction}
\label{sec:intro}

Networked embedded systems, or sensor networks, are finding ubiquitous
application in densely sampling phenomena --- from structural
properties of buildings to wildlife behavior --- that were previously
difficult or impossible to observe.  Like embedded systems, sensor
networks operate in resource and energy constrained environments with
minimal operating system support.  However, unlike traditional
``dedicated'' embedded applications, sensor network applications are
reprogrammable, and applications may change in the field.  For this
reason, sensor network applications can benefit from operating system
support for general-purpose system abstractions.

A useful class of system abstractions provides facilities to
dynamically manipulate system resources.  
%
% Another important class of abstractions are those that facilitate
% changing the owner of a resource.
%
Such abstractions can simplify application development and allow an
application to naturally and efficiently respond to the changing needs of its
environment.  
%
% Several platforms for sensor networks provide a form of dynamic
% resource management.  
%
For example, TinyOS~\cite{TinyOS} uses a buffer-swapping protocol 
between components for
efficient sharing of statically allocated buffers.  
The SOS operating system~\cite{sos} supports dynamic
allocation of memory, while MANTIS OS~\cite{abrach03mantis} supports
both dynamic memory allocation and thread creation.
Frameworks built on
top of sensor network systems also include components that make
resources dynamically available to other parts of the system.  The
VanGo framework's buffer pool for
TinyOS is an example of this form of dynamic resource 
management~\cite{greenstein05vango}.

%
% NOTE: With the addition of buffer swapping in TinyOS added above, is
% this paragraph still needed?
%
% While especially important for the proper management of dynamic
% resources, this kind of ownership model is also useful for managing
% static resources.  For example, TinyOS lacks support for dynamic
% memory allocation, but statically allocated buffers are often
% transferred among components for efficiency reasons.  TinyOS employs a
% split-phase approach to message passing~\cite{TinyOS}: a message
% pointer is passed via a call to {\tt Send.send}, and the sender
% receives the {\tt Send.sendDone} event when message transmission has
% completed.  The calling component should not modify the message
% pointer in any way until receiving the {\tt sendDone} event.
% Ownership is a natural protocol for guaranteeing this behavior:  the
% calling component transfers ownership of the pointer to the callee
% upon a {\tt send} call and regains ownership upon the {\tt sendDone}
% event~\cite{ownershipthread}.


While the ability to manipulate resources increases the expressiveness
of sensor network applications, this expressiveness can be a
double-edged sword.  Improper management of resources can lead to
subtle errors that can affect both the correctness and efficiency of
applications.

For example, consider the architecture of a simple sensor-network
sense and send application.
%, shown in Figure~\ref{fig:surge-dataflow}.
Like many sensor-network applications, this application is
arranged in a dataflow architecture:  raw sensed data captured at the
sensors moves through various filters
% (via the {\tt surge} module)
before being forwarded to a base station.
% (via the {\tt tree routing} module).  
In order to naturally and efficiently implement the message passing,
data buffers are %  dynamically allocated as necessary and 
passed by
reference between system modules rather than by copying.

If incorrectly implemented, this style of data sharing can lead to
serious errors.  First, a module may access a (dangling) reference to
data that has been passed on and possibly freed downstream. 
%
The low-end processor hardware used in current sensor nodes does not
have a memory management unit.  Thus existing sensor node operating
systems for these platforms, like TinyOS~\cite{TinyOS} and
SOS~\cite{sos}, do not support virtual memory, and a bad dereference
can crash the system. 
%
Second, a module may get a data buffer from an upstream module but
forget to either dispose it or pass it on to the next stage for
processing.  Since expensive garbage-collection mechanisms are not
available, such resource leaks will very soon exhaust available
memory.


Our key insight is that data transfer between
modules in sensor-network applications 
typically does not lead to sharing, but instead follows a
producer-consumer pattern.  % Therefore, sensor network programmers
% typically implement correct and efficient resource management using
% implicit {\em ownership-based} data access protocols. 
In this style,
every resource has exactly one {\em owner} module at any point.  The
module that creates the resource (dynamically 
through a kernel call or statically
through a variable declaration) assumes initial ownership.
% Ownership of a resource is explicitly transferred through kernel calls
% that implement message passing.  
% %
% \mynote{ROY: This is a little awkward.  I added the notion of
% ``static'' resource allocation above.  Here I am trying to explain
% that it is valid for a module to simply ``remember'' a resource (ie.
% not leak it).  In the prior version we simply said that a module must
% free or transfer ownership of data.}
% %
Each module has the right to access the resources that it owns but
also the responsibility to eventually either free the resource or transfer
ownership to another module.  
%
A module may not manipulate resources that it does not own.  If all
modules obey this protocol, then there can be neither accesses to
dangling resources nor resource leaks.



Current sensor-network systems do not help programmers to ensure that
this ownership protocol is properly obeyed.  In most systems, this
protocol is completely implicit and can be expressed only informally
through programming conventions and comments.  The SOS operating
system includes an explicit API for expressing ownership
relationships.  However, programmer-declared ownership relationships
are trusted rather than checked, so programmers must take responsibility for
ensuring that this API is properly used.  Our experience has been that
programmers often make subtle mistakes in ownership,
leading to critical and difficult-to-track errors.  

In this paper, we present an approach for automated validation of dynamic
resource management in sensor-network software.  Programmers employ an
explicit API along with lightweight program annotations to express
ownership intentions for resources.  An automated program checker employs these
annotations to analyze each module {\em
at compile time} for violations of the ownership protocol, providing
early feedback to programmers about the potential for dynamic resource
errors.  The checker requires a novel suite of dataflow analyses to ensure
the key ownership invariants:  each resource has a unique owner; each
resource is only manipulated by its owner; and each resource is either
freed by its owner or transferred to another owner.  With our approach,
sensor-network programmers can make use of the expressiveness of
dynamic resource management while retaining confidence in the
reliability of their applications.

We have instantiated our approach in Lighthouse, a tool for statically ensuring
proper management of dynamic memory in the SOS operating
system~\cite{sos}. 
%which employs the C programming language.  
We focus on SOS memory management because of our experience with
memory errors as users of SOS, which was our initial motivation.
However, the underlying dataflow analyses are generic, and our tool is
parametrized by the API for resource creation, ownership transfer, and
resource deletion.  Therefore, we believe that our tool can be easily
adapted both to statically track ownership for other kinds of
resources and to handle C-based operating systems other than SOS.

We evaluated our tool on the historic versions of all user modules in
the SOS CVS repository, as well as on the SOS kernel.  Overall, we ran
the checker on about 40,000 lines of code.  The tool identified 25
suspect memory operations of which eight were actual memory errors in the
historic versions of the user modules, as well as 35 suspect memory
operations of which two were actual memory errors in the SOS kernel.
The latter finding is somewhat surprising, since the SOS kernel code
is relatively mature and was written by systems programming experts.
In summary, our experiments illustrate the practicality of our
approach for providing early feedback about memory errors in real
sensor-network software.

\mynote{Update appropriately once the other sections have been finalized.}
The rest of the paper is structured as follows.
Section~\ref{sec:example} illustrates the problems for resource
management via examples from TinyOS and SOS.  It continues
with an informal description of the ownership protocol that resolves
these problems.  Section~\ref{sec:alg} presents a precise description
of this protocol and the way in which our tool enforces it.
Section~\ref{sec:eval} provides our experimental results, which
illustrate the utility of our tool in practice.
Section~\ref{sec:related} compares against related work, and
Section~\ref{sec:conc} concludes.

