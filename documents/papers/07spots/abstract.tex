% -*- tex-main-file: "main.tex" -*-

\begin{abstract}
%
\noindent Many sensor network systems expose general interfaces to
system developers for dynamically creating and/or manipulating resources
of various kinds.
%
While these interfaces allow programmers to accomplish common system
tasks simply and efficiently, they also admit the potential for
programmers to mismanage resources, for example through leaked
resources or improper resource sharing.  These kinds of errors are
particularly problematic for sensor networks, given the resource
constraints and lack of memory protection on current sensor platforms.
%

We describe a static analysis that brings the
safety of static resource management to systems that dynamically
manage resources.  Our analysis is based on the observation that
sensor network applications often manipulate resources in a
producer-consumer pattern.  In this style, each resource has a unique
{\em owner} component 
at any given point in time, who has both the sole capability to
manipulate the resource and the responsibility to properly dispose of
the resource or transfer ownership to another component.
Our analysis enforces this ownership discipline on components at
compile time.

We have instantiated our approach as a 
tool to ensure proper management of dynamically allocated
memory in programs written on top of SOS, a sensor network operating
system.
%
We have evaluated the tool on all historical versions of all user
modules in the SOS CVS repository, as well as on the SOS kernel
itself.  
%
Our tool generated
%
25 warnings of which 8 were real errors when checking user modules
and 35 warnings of which 2 were real errors when checking the kernel,
% 
demonstrating the practical utility of our approach for 
sensor network systems.
%
\end{abstract}
