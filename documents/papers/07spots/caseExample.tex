\section{Resource Management and Ownership}
\label{sec:example}

In this section we illustrate the benefits and pitfalls of dynamic
resource management for sensor-network applications through two
examples, from TinyOS and SOS respectively.  We then demonstrate how
an ownership discipline naturally captures the requirements for proper
resource management in these systems.

\subsection{Buffer Swapping in TinyOS}

\begin{figure}[t]
\input{GenericBaseM.c}
\caption{TinyOS implementation of {\tt GenericBaseM receive
interface}\label{fig:genericbase}}
\end{figure}

\mynote{Make sure all of this is true!}
Applications in TinyOS typically receive data from the network stack
by implementing the {\tt receive} event handler from the
\mynote{INTERFACE} interface.  This handler is passed a
\code{TOS\_MsgPtr}, which is a pointer to the incoming data, as well
as a flag describing the data source.  Passing a pointer 
obviates the need for copying, 
allowing the receiver to naturally and efficiently access the data.

To obtain the benefits of passing data by reference without resorting
to dynamic memory allocation, TinyOS commonly
employs a {\em buffer-swapping} protocol.  Under this protocol the
{\tt receive} handler must return a \code{TOS\_MsgPtr} that points to
a free buffer, thereby replacing the buffer passed to it by the caller.
In this way, the network layer can pass buffers by reference to its
clients without running out of memory.

As an example, 
Figure~\ref{fig:genericbase} shows 
the {\tt receive} event handler in {\tt
GenericBaseM}, an application that uses a sensor node as a bridge
between a base station and the rest of a sensor network.  
If there are no messages waiting to be sent (line 6), the handler
forwards the message out over the bridged interface.  It also sets
{\tt nextReceiveBuffer}, which is returned at line 29, to a free
message buffer {\tt ourBuffer}.  If a message is currently pending,
the handler simply ignores the incoming message.  In that case, {\tt
  nextReceiveBuffer} points to the incoming buffer (line 02), which
is therefore returned to the caller to be reused.

While the code in Figure~\ref{fig:genericbase} properly implements the
buffer-swapping protocol, this is not necessarily obvious, due to the
dependence on which path is taken through the conditional at line 6.  
Further, small changes to the code can easily violate the
protocol and cause critical application errors.  For example, suppose
the programmer failed to update \code{nextReceiveBuffer} at line 10.
In that case, the handler would be simultaneously forwarding the
received message pointer over the bridged interface and returning it to the
caller for reuse.  If the caller reuses the buffer while it is still
being accessed by the message receiver, undefined behavior will
result.
As another example, suppose the programmer failed to update
\code{ourBuffer} at line 11.  In that case, whenever the \code{if}
branch is taken, the \code{received} buffer would be leaked.


% The \code{receive} interface implemented in {\tt GenericBaseM}, and in
% other end user components that receive messages, depends on proper use
% of buffer swapping.  Problems arise if a component attempts to store a
% pointer to the incoming buffer for latter access and then returns a
% pointer to that same buffer, as would be the case if if line 10 were
% left out of figure~\ref{genericbase}.  In this situation two different
% components now have references to a buffer that each component believes
% it has exclusive access to.  Or if a TinyOS component should return a
% fresh buffer and fail to create a persistent reference to the incoming
% buffer, buffers would be leaked.  In figure~\ref{genericbase} this
% could happen if line 02 were omitted.  Added confusion arises in
% more complex code, such as the \code{if} statement at line 06 in {\tt
% GenericBaseM}, that returns different buffers depending on specific
% conditions. 


% A key property demonstrated by this TinyOS component is that of buffer
% swapping.  Buffer swapping is a common technique used within TinyOS to
% allow data to quickly pass through stacks of components.  Rather than
% dynamically allocating a new buffer, a call stack can simply return a
% free buffer (out of a pool of statically allocated buffers) to the
% caller.  In the TinyOS environment, handlers that implement buffer
% swapping have implicit memory ownership responsibilities.  The handler
% can be thought of needing to claim the incoming \code{TOS\_MsgPtr} and
% release a buffer to the caller.  Note that this can be accomplished by
% returning the incoming buffer.


\subsection{Dynamic Memory Allocation in SOS}

\begin{figure}[t]
\input{surge.c}
\caption{SOS implementation of {\tt surge}\label{fig:surge}}
\end{figure}

SOS is an operating system for sensor networks that 
supports dynamically loaded and unloaded software modules.
The SOS kernel provides an API for these modules to dynamically
allocate memory, which can simplify applications and allow them to
make more efficient usage of memory.
%As shown in Figure~\ref{fig:surge}, 
The SOS functions \code{ker\_malloc} and \code{ker\_free} are similar
to their C namesakes, respectively
allocating and disposing of memory.  
As in C, programmers are responsible for manually managing their
dynamic memory.  

Figure~\ref{fig:surge} shows a portion of the SOS module that
implements {\tt surge}, a simple application that takes
sensor readings and sends the readings over a multihop network to a
base station~\cite{nesC}.  SOS modules employ an event-driven style.  
The {\tt surge} module's persistent state (which is maintained by the
kernel) and an
incoming message are passed to the \code{surge\_module} function,
which is the module's top-level event handler.  The handler
uses a \code{switch} statement to dispatch based on the kind of
message received.  
When sensor data becomes available at a node
(\code{MSG\_DATA\_READY}), the handler dynamically 
allocates a new message (line
9) to hold the data and sends it out via a tree-routing mechanism
(line 12).  % The \code{SOS\_MSG\_RELEASE} flag on this
% \code{post\_long} call indicates that the current module releases
% ownership of \code{pkt}.
When the data arrives at the root of the
routing tree
(\code{MSG\_TR\_Data\_PKT}), the module % takes ownership of the data
% via the kernel call \code{ker\_msg\_take\_data}.  (If the sender did
% not release ownership, then the kernel allocates a new copy of the
% data for the receiver.)  It then 
forwards the
message over the network.

The code in Figure~\ref{fig:surge} properly manages dynamic memory via the
SOS kernel's API.  However, this fact is
not at all obvious.  For example, the module depends on
the assumption that the memory
allocated on line 9 will eventually be freed downstream by the network layer,
which is passed the memory at line 21.  \mynote{Is that true?}  
As with the earlier TinyOS example, small
changes to this module can also cause critical errors.  For example,
the programmer must be sure not to access \code{payload} after it is sent
out via
\code{post\_net} in line 21.  Otherwise, a dangling pointer error may
result, since the network layer may have already freed the buffer. 

% as it depends on a complex relationship between the code's 
% uses of
% \code{ker\_malloc} and \code{ker\_msg\_take\_data} for acquiring memory
% and its uses of \code{ker\_free} and \code{SOS\_MSG\_RELEASE} for
% releasing memory.  As in the TinyOS example above, small changes to
% the code can cause critical errors.  For example, suppose the
% programmer fails to take ownership of the incoming data at line 20.
% Assuming the data was released by the sender, the SOS kernel will then
% automatically free this memory when the handler returns, causing a
% dangling pointer access when the data is sent through the network
% layer.  Similarly, a dangling pointer error is possible by 
% accessing memory after releasing it (e.g., accessing \code{pkt} after
% \code{post\_long} in line 12), since the memory may have been freed by
% the kernel or the new owner.
% Finally, omitting the \code{SOS\_MSG\_RELEASE} flag at line 14 will
% cause a memory leak, since the receiving module will attempt to take
% ownership and thereby acquire a new copy of the message.
 
% Because this can be tedious and error prone, the
% SOS kernel tracks memory {\em ownership} in order to provide a simple
% form of garbage collection.  A module acquires initial
% ownership of memory that it allocates via \code{ker\_malloc}.  The SOS
% kernel also provides an API that allows modules to release ownership
% of data that it sends and take ownership of data that it receives.
% The kernel automatically garbage-collects all memory owned by a module when that
% module is unloaded, as well as all memory whose ownership is 
% released by a sender but not
% taken by the corresponding receiver.  

% SOS development runs into similar difficulties with its dynamic
% memory.  Leaking memory is a problem in all systems that allow dynamic
% memory creation.  This would occur in {\tt surge} if the
% \code{SOS\_MSG\_RELEASE} flag were left out of the call to
% \code{post\_net} on line 23 of figure~\ref{surge}.  Further, tricky
% dangling pointer would result from forgetting to claim ownership of
% the incoming data before it is sent out over the UART at line 20 of
% figure~\ref{surge}.

% The message receiver can explicitly take
% ownership via a kernel call \code{ker\_msg\_take\_data}; otherwise the
% kernel will free the memory after the 
% message to be created with the most recent sensor data, encapsulated,
% and sent out via a tree routing mechanism.  The
% \code{MSG\_TR\_DATA\_PKT} handler is executed at the root of a data
% routing tree when a data message arrives, and takes control of the
% incoming encapsulated message so that it can be forwarded over a UART.

%%
% ROY: 11/7/06 I condensed the following three paragarphs into the
% more concise description provided above.
%%

% The \code{surge\_module} function is the
% entry point into {\tt surge} for messages from the kernel and from
% other modules.  The function takes two arguments: a pointer to the
% module's persistent state, which is saved in the kernel, and a pointer
% to the current message.  A {\tt switch} statement is used to direct
% each message type to an appropriate handler.  The handlers of interest
% in this example are for the messages types {\tt MSG\_DATA\_READY} and
% {\tt MSG\_TR\_DATA\_PKT}.
% 
% A sensor sends the message {\tt MSG\_DATA\_READY} to the surge module
% when requested sensor data is ready to be read. The sensor data is
% passed as the {\tt data} field of the message, which in general always
% contains a message's payload.  Upon receiving this message, the {\tt
% surge} message handler allocates a new packet ({\tt ker\_malloc}) to
% be sent to the base station and posts a message ({\tt post\_long}) to
% the tree-routing module in order to forward the sensor data.  The {\tt
% post\_long} call is asynchronous, causing the kernel to package up all
% the given arguments into a {\tt Message} structure and to schedule
% this message for eventual delivery.
% 
% The message {\tt MSG\_TR\_DATA\_PKT} is sent by the tree-routing
% module when data is received at the base station node.  Upon receiving
% this message, the {\tt surge} message handler confirms that the
% current node is the base station.  If so, the message handler forwards
% the data to the UART driver via an asynchronous message send ({\tt
% post\_net}).


%%%%%%%%%%%%%%%%%%%%%%%%%%%%%%%%%%%%%%%%%%%%%%%%%%%%%%%%%%%%
%%%%%%%%%%%%%%%%%%%%%%%%%%%%%%%%%%%%%%%%%%%%%%%%%%%%%%%%%%%%
%%%%%%%%%%%%%%%%%%%%%%%%%%%%%%%%%%%%%%%%%%%%%%%%%%%%%%%%%%%%


% \subsection{Improper Resource Management}

% The ability to manipulate resources demonstrated above results in
% small and efficient programs well suited for the embedded processors
% used within sensor networks.  However, improper resource management
% can cause serious problems for sensor network applications.


% \smallskip\noindent{\bf Potential problems with {\tt GenericBaseM}}

% The \code{receive} interface implemented in {\tt GenericBaseM}, and in
% other end user components that receive messages, depends on proper use
% of buffer swapping.  Problems arise if a component attempts to store a
% pointer to the incoming buffer for latter access and then returns a
% pointer to that same buffer, as would be the case if if line 10 were
% left out of figure~\ref{genericbase}.  In this situation two different
% components now have references to a buffer that each component believes
% it has exclusive access to.  Or if a TinyOS component should return a
% fresh buffer and fail to create a persistent reference to the incoming
% buffer, buffers would be leaked.  In figure~\ref{genericbase} this
% could happen if line 02 were omitted.  Added confusion arises in
% more complex code, such as the \code{if} statement at line 06 in {\tt
% GenericBaseM}, that returns different buffers depending on specific
% conditions. 


% \smallskip\noindent{\bf Incorrect versions of {\tt surge}}

% SOS development runs into similar difficulties with its dynamic
% memory.  Leaking memory is a problem in all systems that allow dynamic
% memory creation.  This would occur in {\tt surge} if the
% \code{SOS\_MSG\_RELEASE} flag were left out of the call to
% \code{post\_net} on line 23 of figure~\ref{surge}.  Further, tricky
% dangling pointer would result from forgetting to claim ownership of
% the incoming data before it is sent out over the UART at line 20 of
% figure~\ref{surge}.


%%%%%%%%%%%%%%%%%%%%%%%%%%%%%%%%%%%%%%%%%%%%%%%%%%%%%%%%%%%%
%%%%%%%%%%%%%%%%%%%%%%%%%%%%%%%%%%%%%%%%%%%%%%%%%%%%%%%%%%%%
%%%%%%%%%%%%%%%%%%%%%%%%%%%%%%%%%%%%%%%%%%%%%%%%%%%%%%%%%%%%

\subsection{Memory Ownership}

Examination of the sensor-network applications 
presented above reveals an underlying
theme.  In both examples, resources are shared dynamically across multiple
software components, to support simple and efficient access to these
resources.  However, only one component requires access to a given
resource at any point in time.  
In other words, proper resource management
naturally obeys an {\em ownership} discipline.  In this
style, each resource has a unique owning component (or {\em owner}), 
who has the sole
capability to access the resource.  The owner also the responsibility to
eventually dispose of the resource or explicitly {\em transfer} 
ownership to another
component.

% In this work, we augment these implicit
% ownership directives to provide a protocol governing memory management
% that is sufficient to ensure the absence of memory errors.  Our
% protocol makes explicit the common programming idiom in sensor-network
% systems, whereby data is rarely shared but instead follows a
% producer/consumer model.  
% %

% If all components obey this ownership discipline, then resource
% management errors are prevented.
% Since only the owner can manipulate a resource,
% ``dangling'' accesses to the resource are avoided.  Since each resource must
% eventually be transferred to another component or freed, resource 
% leaks are avoided.  
%

In this paper, we describe an approach for statically enforcing an
ownership discipline on sensor-network applications.  Lightweight
annotations allow programmers to specify the ownership transfer
requirements of each function.  For example, in
Figure~\ref{fig:genericbase} the \code{receive} function would be
annotated to indicate that ownership of the \code{received} parameter
is being transferred from the caller to the callee, and that ownership
of the return value is being transferred from the callee to the
caller.  Given these annotations, we define a static checker that can
analyze each function of a component in isolation, providing early and
modular feedback on the potential for resource management errors.

Informally, our static ownership discipline
enforces the following rules:
%
\begin{itemize}
%
\item Each function may only refer to dynamic memory that it owns.  This
  includes memory allocated in the function, memory in the surrounding
  component's persistent store, and any formal parameters annotated as
  having their ownership transferred to the component.
%
\item Each function that allocates or obtains ownership of a block of
  dynamic memory 
must eventually free that memory, transfer its ownership to another 
component, or store the memory in a persistent location.
%
\item After a function frees or releases memory that it owns, 
it may no longer access the memory.
%
\end{itemize}

Static checking is necessarily an approximation of the runtime
requirements for proper resource management.  As described in the next
section, the rules above can lead to both false positives and false
negatives.  However, our experimental results indicate that the rules
are a practical approach for detecting resource management errors in
real sensor-network software.


% \smallskip\noindent{\bf Applying Ownership Protocol to {\tt GenericBaseM}}
% Revisiting the TinyOS example above, we can view buffer swapping as a
% form of mutual ownership transfer.  The caller of \code{receive}
% should pass a buffer that it owns, transfering ownership to the
% callee.  The callee in turn should return a buffer that it owns,
% transfering ownership to the caller.  

\subsection{TinyOS and SOS, Revisited}

The \code{reserve}
implementation in Figure~\ref{fig:genericbase} obeys our static ownership
discipline, assuming the annotations described above. 
Upon entry to the
\code{receive} event handler the \code{GenericBaseM} component owns the
buffer referenced by the global variable \code{ourBuffer} and gains
ownership of the \code{received} buffer passed into the function.
Within the body of the function, these are the only two
buffers accessed, although they are at times accessed via the alias
\code{nextReceiveBuffer}.  If the message is able to be sent over the
bridged interface, the function returns the buffer originally referenced by
\code{ourBuffer} and retains ownership of \code{received} by storing
it in \code{ourBuffer}.  
If the message is unable to be sent, the function
returns the \code{received} buffer, thereby giving up ownership and
all references to the buffer.
\mynote{Seems like we need to say something about the send calls.  If
  ownership is not transferred, our rules would not allow the
  recipient to access the sent buffer!  So what's the deal?}

The ownership discipline also catches the potential errors that can be
introduced into \code{GenericBaseM}, as described earlier.  For
example, suppose the programmer failed to update \code{nextReceiveBuffer}
at line 10, so that \code{received} is returned to the caller even
when it is sent over the bridged interface.  In this case, the
\code{receive} function must not manipulate \code{received} after the
return, since the function transfers ownership to the caller at that
point.  This rule is violated by the asynchronous \code{send} calls,
which may not execute until after the \code{receive} function has
completed.  \mynote{Not clear that our checker catches this.}
Suppose instead that the programmer failed to update \code{ourBuffer}
at line 11, leading to a memory leak.  In this case, the
\code{receive} function fails to either free, transfer, or
persistently store \code{received}, violating our second rule above.
 
% protocol helps us to reason about potential bugs that
% could be introduced into \code{GenericBaseM}.  As noted above, omitting
% line 11 from the implementation in figure~\ref{genericbase} results in
% a resource leak.  However, this violates the resource ownership
% protocol.  Since the \code{receive} interface owns the data referenced
% by the formal variable \code{received}, a violation results since that
% data is not persistently stored or released by the interface when the
% \code{if} clause if the \code{if} statement is taken.

% \smallskip\noindent{\bf Applying Ownership Protocol to {\tt surge}}


The SOS kernel already dynamically tracks memory
ownership and has an associated API for ownership transfer, which
inspired our work.  For example, the \code{SOS\_MSG\_RELEASE} flag in
the call to \code{post\_long} at
line 14 in Figure~\ref{fig:surge} indicates that ownership of
\code{pkt} should be transferred to the callee (or freed by the kernel
if the callee does not explicitly take ownership).  Similarly, the
call to \code{ker\_msg\_take\_data} at line 20 allows the caller to
take ownership of \code{msg} under the alias \code{payload} (or to
take ownership of a copy of \code{msg}, if the message provider did
not explicitly release it).  While useful as documentation, usage of
this API is completely unchecked in SOS, so programmers must manually
ensure adherence to the implicit ownership protocol associated with
the API.  Our work makes this protocol explicit and provides static
checking for conformance.

Given this API, the {\tt surge} module in figure~\ref{fig:surge} can be seen to
adhere to our static ownership discipline.  
The \code{MSG\_DATA\_RDY} message handler allocates \code{pkt},
thereby taking initial 
ownership. This pointer is then dereferenced in order to provide the
sensor data to be sent up the routing tree.  This pointer manipulation
is safe since the module has ownership.  The module then releases
ownership by posting \code{pkt} to the tree routing module using the
\code{SOS\_MSG\_RELEASE} tag.  After this release, the module does not
access \code{pkt} again. 
The handler for \code{MSG\_TR\_DATA\_PKT} also conforms to the
protocol.   When the current node is the base station, the handler
explicitly acquires ownership of the message's data using
\code{ker\_msg\_take\_data}.  This allows the module to manipulate the data
and to pass it to the UART.  The \code{post\_net} call explicitly
releases the data, fulfilling the module's obligation to that data.
After the release, the data is no longer accessed.

Again, our static rules would catch changes to this code that cause
dynamic resource management errors.
For example, suppose the branch for
\code{MSG\_DATA\_READY} did not release ownership of \code{pkt} by
setting the \code{SOS\_MSG\_RELEASE} flag in the call to {\tt
post\_long}, thereby retaining ownership but leaking the memory.    
In this case,
the code fails to either free, transfer, or
persistently store \code{pkt}, violating our second rule above.  On
the other hand, if the \code{SOS\_MSG\_RELEASE} flat is set but the
code accesses \code{pkt} after the call to \code{post\_long}, then the
dangling access is caught by our third rule.
 

%%
% ROY: 11/07/06 Commenting this out.  I do not think that the end user
% needs these SOS specfic details to understand this work.
%%

% Transfer of dynamic memory ownership occurs at the end points of a
% message.  First, the owner of a block of dynamically allocated memory
% can explicitly {\em release} ownership of that block when it is passed
% as the payload in a message.  This is accomplished by setting the
% \texttt{SOS\_MSG\_RELEASE} flag in the corresponding {\tt post\_*}
% call.  For example, the {\tt surge} module releases ownership of the
% newly allocated {\tt pkt} upon sending it to the tree-routing module.
% Second, a module can acquire ownership of a message's payload, which
% is stored in the {\tt data} field, by calling
% \texttt{ker\_msg\_take\_data} on an incoming message.  The function
% returns a pointer to the message's payload.  For example, if the
% current node is the base station, the {\tt surge} module explicitly
% takes ownership of the given message's data under the name {\tt
% payload}.
% 
% There are four release/take scenarios to consider.  If data is both
% released by its sender and taken by its receiver, then ownership of
% the data is transferred from the sender to the receiver.  If data is
% released by its sender but not taken by its receiver, then the kernel
% automatically frees the memory after the receiver's message handler
% completes.  If data is not released by its sender but is taken by its
% receiver, then the sender keeps ownership of the original message and
% the receiver gains ownership of a new block of memory containing a
% copy of that data.  Finally, if the data is not released by the sender
% and not claimed by the receiver, then the sender keeps ownership of
% the original message and the receiver has direct access to
% ``borrowed'' data for a limited period of time.  This last case is not
% generally used in SOS due to the synchronization complications that
% can result.

%%
% ROY: 10/18/06 This is not adding much to the discussion.
%%

% In order to provide a simple form of automatic garbage collection for
% dynamically allocated memory, the SOS kernel imposes an {\em
% ownership} model on dynamic memory~\cite{sos}.  Each block of memory
% has a unique owner at any point in time, and the kernel maintains a
% mapping from each block of memory to its owner.  A block's initial
% owner is the module that allocates that block.  For example, the call
% to {\tt ker\_malloc} sets the {\tt surge} module as the initial owner
% of the newly allocated block.  When a module is removed from the
% system at run time, the kernel automatically frees all memory owned by
% that module.
%
% Simple garbage collection introduces the potential for more
% dangling pointer errors, since the removal of a module implicitly
% frees the memory it owns, even if other modules have pointers to
% that memory.


