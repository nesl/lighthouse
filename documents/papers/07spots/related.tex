\section{Related Work}
\label{sec:related}


As mentioned earlier, the memory manager in the SOS operating
system~\cite{sos} includes ownership annotations that are used by the kernel
to dynamically track memory ownership.  When a module is removed from the
running system, the kernel automatically deallocates all the memory that the
module owns.  However, ownership annotations in SOS are {\em trusted}.  For
example, missing annotations can cause the kernel's information to become out
of date.  Also, there are no checks that owners and nonowners meet their
obligations.  For example, nothing prevents a module from allocating memory
and neglecting to eventually free it.  Our work grew out of a desire to
prevent these kinds of errors from occurring in SOS applications.


Other sensor network platforms have included support for static
checking to prevent other classes of errors.  For example, the
nesC~\cite{nesC} language for sensor networks employs a whole-program
analysis to statically detect race conditions.  As another example,
the galsC~\cite{TinyGALS, galsC} language for embedded systems employs
an analysis to ensure the type safety of connections between
components.  More recent work on Safe TinyOS and
UTOS~\cite{regehr06memory} combines static analysis with run time
protection of memory within sensor network systems.


A complementary approach to improving the reliability of sensor
network software is through new language abstractions.  For example,
researchers have explored language support for component-based
programming~\cite{TinyOS,nesC,galsC}, region
abstractions~\cite{conf/mobisys/WhitehouseSCB04,conf/nsdi/WelshM04},
component composition~\cite{conf/sensys/GreensteinKE04}, and
programming in the aggregate~\cite{1052213,conf/dcoss/GummadiGG05}.
New language constructs enable easier expression of certain
programming idioms.  They can also make programs more amenable to
static checking.  Our tool currently analyzes ordinary C programs,
since that is the language that SOS employs, but it would be
interesting to explore ways to leverage specialized language
constructs to improve the tool's effectiveness.


Recent work in the programming languages community has explored the
concept of {\em ownership
types}~\cite{ownership,ownership2,BoyapatiEtAl02,aliasjava} for
object-oriented languages.  Ownership types designate an owner object
for each object, and the static type system ensures a form of {\em
confinement} for each object with respect to its owner.  For example,
a typical invariant guaranteed by ownership type systems is that an
object will only be accessed by its owner or by other objects owned by
the same owner.  In this way, an object's owner forms a dynamic scope
within which the object is confined.  Related work on {\em confined
types}~\cite{confined1,confined2} provides a more static form of
confinement, in which an object is guaranteed not to escape a
particular static scope.


Our work provides a static notion of ownership analogous to that of
confined types:  all accesses to a given resource may only occur
within the static scope of its owning module.  On a technical level,
however, the foundation of our work is quite distinct from that of
both ownership types and confined types, as we rely on dataflow
analysis rather than on type systems.  Our use of dataflow analysis is
necessary in order to safely accommodate dynamic transfer of
ownership, which the systems described above lack.  Ownership transfer
is critical in practice for sensor network applications, for example
it is needed 
to properly account for split-phase operations.  Although recent work
has explored a form of transfer in the context of ownership type
systems~\cite{DBLP:conf/ecoop/BanerjeeN05}, that work requires
programmers to provide detailed assertions about ownership, and these
assertions are proven as part of a more general program specification
and verification framework.


Finally, there have been several proposals for a form of {\em unique}
or {\em linear}
pointer~\cite{Boyland:2001:ABU,aliasjava,Wad90:linear,adoption-focus},
which is guaranteed to be the only reference to its referent.  These
systems typically include a form of transfer of uniqueness from one
pointer to another.  For example, a {\tt unique} pointer in
AliasJava~\cite{aliasjava} can be transferred as long as a dataflow
analysis shows that the original pointer is no longer accessed after
the transfer.  Our tool allows a resource to have any number of
aliases from within its owner, which is less restrictive than the
uniqueness requirement.  At the same time, transfer is still allowed
safely, as long as none of these aliases are accessed after the
transfer.

