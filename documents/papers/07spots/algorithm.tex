\section{Analysis Specification and Implementation}
\label{sec:alg}


\subsection{Specification}

We now formalize the logical
specification of our ownership-based resource management protocol.
For simplicity, we specify the protocol on an idealized imperative
language that includes the fundamental operations for resource
management.  The language's atomic statements are as follows, where
{\tt x} and {\tt y} range over {\em resource handle names}:
\[
y := x\mid \mathrm{alloc}\ x \mid \mathrm{free} \ x \mid 
  \mathrm{get}\ x \mid  \mathrm{store}\ x 
%
\]
Programs are constructed by sequencing statements of the above form,
and we assume the presence of standard control-flow constructs
including conditionals and loops.

Our language's statements
abstractly capture resource-related program operations and
have the following informal semantics.
An assignment $y :=  x$ assigns the resource referenced by handle $x$ to the
handle $y$.
The operation $\mathrm{alloc}\ x$ allocates a new resource and returns a handle
$x$ to the resource.
The operation $\mathrm{free}\ x$ frees the resource
referenced by the handle $x$.
In addition, we assume the presence of a persistent store that holds a
single resource.  The operation $\mathrm{store}\ x$ puts the resource
referenced by handle $x$ into the store, and the operation
$\mathrm{get}\ x$ assigns the resource contained in the persistent store
to $x$.

Given this core language, we formally specify rules for proper
resource usage as 
invariants on each dynamic program execution.  A natural formalism for
these invariants is
{\em linear temporal logic} (LTL)~\cite{Emerson90,MannaPnueli92vol1},
which we briefly review.
%% Technically, the specifications are in first order LTL, since
%% they quantify over resource handles, but for a fixed program, these are constants. 
An LTL formula is constructed from atomic predicates using boolean
operations and the {\em temporal operators} $\Box$, $\Diamond$, 
and $\Next$.
An LTL formula is evaluated at a given execution state
with respect to a fixed execution path through the program.
The formula $\Box f$ holds at state $s$
if the formula $f$ holds on $s$ and each subsequent state in
the path.  
The formula $\Diamond f$ 
holds at state $s$ if the formula $f$ holds on $s$ or some
subsequent state in the path.  
The formula $\Next f$ holds at state
$s$ if the formula $f$ holds on the successor state of $s$ along the
path.


For a handle $x$, we write $\access(x)$ for any operation that syntactically
accesses the resource referenced by $x$, that is, one of $\mathrm{free}\ x$,
$\mathrm{store}\ x$, or an assignment with $x$ on the right-hand side.
We also introduce a relation $\alias(x,y)$ on pairs of handles:
$\alias(x,y)$ holds at an execution state
iff handles $x$ and $y$ refer to the same
resource in that state. 
%% The relation can be statically approximated as an overapproximation
%% $\mayalias(x,y)$ that holds
%% if $x$ may be aliased to $y$, and an underapproximation
%% $\mustalias(x,y)$ that holds if $x$ is definitely
%% aliased to $y$ \cite{???}.
We now define two key correctness invariants for proper resource
management. 

%%%%%%%%%%%%%%%%%%%%%%%%%%%%%%%%%%%%%%%%%%%%%%%%%%%%%%%%%%%%
\smallskip
\noindent
{\bf No Leaks.} The first property formalizes the absence of resource
leaks by enforcing that along every program execution path, every allocation
$\mathrm{alloc}\ x$ is necessarily followed by either
a free or a store of a handle
that refers to the same resource as $x$:
\[
\Box \left( (\mathrm{alloc}\ x) \rightarrow
	\Diamond \left( (\mathrm{free}\ y 
                              \vee
                        \mathrm{store}\ y) \wedge \alias(x,y) 
                 \right) \right)
\]
%% In our tool, we use the static underapproximation $\mustalias$ for the alias
%% relation. 
%\begin{displaymath}
%%
%\forall \ x, \ t_{i} = \mathrm{malloc} \ x \ 
%%
%\Rightarrow \
%%
%\exists \ j > i \ 
%%
%\mathrm{st.} \
%%
%t_{j} = \mathrm{free} \ [\mathrm{must \ alias} \ x] \ 
%%
%\vee \ t_{j} = \mathrm{store} \ [\mathrm{must \ alias} \ x] 
%\end{displaymath}
%

%%%%%%%%%%%%%%%%%%%%%%%%%%%%%%%%%%%%%%%%
\smallskip\noindent
{\bf No Dangling Pointers.} The second
property states that once a resource has
been freed, it is never accessed again:
%% Formally, upon an operation $\mathrm{free}\ x$, there should not be any 
%% subsequent access to the resource that was freed:
\[
\Box ( \mathrm{free}\ x \rightarrow \Next (\Box (\lnot (\access(y) \wedge \alias(x,y))))) 
\]
%% In our tool, we approximate $\alias$ with the statically computed overapproximation
%% $\mayalias$.

%\begin{displaymath}
%%
%\forall i, \ t_{i} = \mathrm{free} \ x \ 
%%
%\Rightarrow \
%%
%\forall \ j > i, \ t_{j} \neq \mathrm{op} \ [\mathrm{may \ alias} \ x] 
%%
%\end{displaymath}

%%%%%%%%%%%%%%%%%%%%%%%%%%%%%%%%%%%%%%%%
%\smallskip
%\noindent
%{\bf Allocation Before Use.}
%The final property requires that any access to a handle is preceded by 
%an allocation or retrieval from the store:
%\[
% \Box (\access(x) \rightarrow \PastDiamond (\alias(x,y) \wedge 
%	\left[ \mathrm{alloc}\ y \vee \mathrm{get}\ y \right] ) )
%\]
%% The $\PastDiamond$ operator is analogous to $\Diamond$ but quantifies
%% over all {\em predecessor} states in the execution path from the given
%% state, rather than successor states.
%\begin{displaymath}
%%
%\forall \ i, \ t_{i} = \mathrm{op} \ x \ 
%%
%\Rightarrow \
%%
%\exists \ j \geq i \ 
%%
%\mathrm{st.} \
%%
%t_{j} = \mathrm{malloc} \ [\mathrm{must \ alias} \ x] \ 
%%
%\vee \ t_{j} = \mathrm{get}_{y} \ [\mathrm{must \ alias} \ x] 
%\end{displaymath}
%
%This property is not verified by the checker.


\subsection{Implementation}

Our tool checks SOS programs at compile time for violations of the
above two properties.
The tool performs a
static dataflow analysis for each property
on a program's control-flow graph (CFG),
which statically represents all possible execution paths.  The
$\mathrm{alloc}$ and $\mathrm{free}$ operations are respectively
represented by {\tt
  ker\_malloc} and {\tt ker\_free}.  To identify a module's persistent
state, we provide a new attribute {\tt sos\_state} for formal
parameters.  For example, the {\tt state} argument to a module's event
handler would be annotated with this attribute.

%% The $\mathrm{get}$
%% and $\mathrm{store}$ operations are respectively represented by accesses and
%% updates to the contents of the {\tt state} argument to a module's
%% event handler.  We also allow programmers to annotate formal
%% parameters with an {\tt sos\_state} attribute, to indicate that it is
%% part of the module's persistent state.  This is useful, for example,
%% when a module's event handler passes the persistent
%% state to a helper function, which happens relatively frequently.

Our dataflow
analyses are implemented in the CIL front end for C~\cite{CIL}, which
parses C code into a simple intermediate format and provides a
framework for performing analyses on the intermediate code.  Each
analysis is {\em modular}, considering each module independently and
analyzing the CFG for each procedure
within the module in isolation.
Care must be taken for our static checker to conservatively approximate the
dynamic conditions that must be satisfied.  Two notable issues are the
treatment of pointer aliasing and of procedure calls, which we discuss
in turn.

First, the invariants described above depend on dynamic alias
information, which cannot be exactly computed at compile time.
Instead, two standard approximations are {\em must-alias} analysis and
{\em may-alias} analysis.
A must-alias analysis underapproximates the
dynamic alias relations:  if a must-alias analysis determines that $x$
and $y$ alias at a particular program point, then
$\alias(x,y)$ definitely holds at any run-time
execution state corresponding to that program point.
A may-alias analysis overapproximates the
dynamic alias relations:  if a may-alias analysis determines that $x$
and $y$ {\em cannot} alias at a particular program point, then
$\alias(x,y)$ definitely does not hold at any run-time
execution state corresponding to that program point.

Our checker requires both kinds of static alias approximations.  In
the first property described above, alias information is
used to ensure that something definitely happens, namely that an
allocated resource is eventually freed.  Therefore, in this case we
approximate the true alias information with must-alias information.
In the second property described above, alias information is used to
ensure that something definitely does not happen, namely an access to
a freed resource.  Therefore, in this case we approximate the true
alias information with may-alias information.

Our implementation currently uses a simple flow-sensitive 
must-alias analysis.  For the may-alias analysis, we
use a fast flow-insensitive alias analysis provided by the CIL
framework.  Static alias analysis can be quite imprecise in the
presence of complex pointer manipulations and pointer structures.  For
example, CIL's may-alias analysis does not distinguish among the
fields of a structure, instead considering them to always potentially
alias one another.  However,
our experimental results indicate that
these limitations do not result in an inordinate number of false
positives for the checker.
This positive result is likely due to
the stylized ways in which dynamic memory is
manipulated in sensor-network applications, which our ownership model
captures well.  

Second, our simple core language used to specify the resource
invariants does not contain procedure calls, which must be properly
handled in our implementation.  As mentioned above, our analyses are
{\em intraprocedural}, meaning that each procedure is checked in
isolation.  To make such checking both correct and precise, we rely on
the ownership attributes {\tt sos\_claim} and {\tt sos\_release}, as
described in the previous section.

Given such attributes, the proper
handling of procedure calls becomes straightforward.  A procedure call
statement 
is treated logically by the checker
as an assignment from actuals to formals, followed by an
assignment from the return value of the call to the left-hand-side
variable (if any).  A formal parameter annotated with the {\tt
  sos\_claim} attribute is treated as an allocation site, just as is
{\tt ker\_malloc}.  A formal parameter annotated with the {\tt
  sos\_release} attribute is treated as a disposal site, just as is
{\tt ker\_free}.

The ability to perform checking modularly allows SOS application
writers to obtain early feedback about the correctness of their
resource management, without requiring access to the rest of the
system.  This is particularly important in a system like SOS, in which
modules can be linked and unlinked dynamically.  In such a setting,
the ``rest'' of the system is a moving target, so it is not really
possible to consider whole-program analysis.

\subsection{Limitations}

As we demonstrate in the next section,
our checker is useful for detecting violations of the ownership
protocol on real sensor-network code.
However, the checker is not guaranteed to find all such violations.
Said another way, the checker can be used for finding memory errors but not
for guaranteeing the absence of all memory errors.  We discuss three
limitations of the checker in this regard.

First, the checker does not precisely handle all of the unsafe features of the
C programming language.  For example, pointer arithmetic is not
statically analyzed.  Instead, an expression of the form $p+i$, where
$p$ is a pointer and $i$ is an integer, is simply treated as if it refers to
the same block of memory as $p$.  If $p+i$ in fact overflows to
another block of memory at run time, the checker's assumption can
cause it to miss errors.
These kinds of limitations are standard for C-based program analyses.

Second, there is a design choice about how to treat messages that a
handler does not explicitly acquire through {\tt
  ker\_msg\_take\_data}.  Technically the ownership protocol 
disallows these messages from being accessed at all, since the
receiving module
is not the owner.  However, enforcing this requirement can cause
spurious errors for patterns of data sharing that are in fact safe,
for example when a module temporarily ``borrows'' data from another
module within a bounded scope.  Therefore, our checker does not
currently enforce the requirement that a module only access memory
that it owns, reducing the number of false positives but also
potentially missing errors.  The checker still ensures that a message
handler does not access memory that has earlier been freed or released.

Finally, our modular checker does not
consider the overall
order in which messages will be received.
However,
there may be application-level protocols that determine how events
are generated in the system, and these protocols can affect the
correctness of ownership tracking.
%
For example, imagine a module that responds to three events:  $\create$
causes the module to allocate a block $b$ and store
it into the module's persistent store,
$\access$ causes the module to access block $b$ via the  store,
and $\delete$ causes the module to deallocate block $b$ via the 
store.
The module properly manages block $b$ as long as the temporal
sequence of events follows the regular expression $\alloc$ $\access^*$
$\delete$.  This ordering ensures that the system never accesses the
block before it is allocated and always eventually deletes the block.

Our tool currently only
tracks data locally within each event handler.
When performing this tracking, the tool assumes on entry
that all data in the persistent store is owned by the module.  The
tool similarly considers a resource to be properly
released when it is placed in the persistent store.  These
assumptions can miss errors, for example if an $\access$ event ever
occurs before the $\alloc$ event.  However, our assumptions provide a
practical point in the design space 
that allows each event handler to be usefully checked
for local violations of the ownership protocol.
%
It will be interesting to extend our system with additional interface
annotations about application-level protocols in the future~\cite{AlurPOPL05,HJM05}.

%% Our checker for correct memory management is based on a 
%% flow-sensitive dataflow analysis.
%% A CIL module called ``ownRelease'' is used to verify that data released by a
%% module is properly treated as dead.
%% %
%% This module first scans the program for locations where a buffer is released.
%% %
%% When such an instruction is found, the module verifies that:
%% %
%% \begin{itemize}
%% %
%% \item No possible predecessor allows the released buffer to escape by storing
%% a persistent alias to the buffer or releasing it to another module.  
%% %
%% Note that this should also check that no predecessor kills the buffer first.
%% %
%% \item No possible successor accesses the buffer in any way.
%% %
%% \end{itemize}


%% This analysis depends on a ``may alias'' analysis of pointers in the checked
%% module.
%% %
%% At this time it accomplishes this via the Ptranal module provided by CIL.


%% The ``ownRelease'' module boot straps allocation by appending a dummy main
%% function to modules that calls the default event handler implemented by the
%% module.
%% %
%% This dummy main function uses a special calls to allocate memory for the
%% module:
%% %
%% \begin{itemize}
%% %
%% \item \texttt{sos\_state\_malloc} is used to specify the creation of the state
%% buffer
%% %
%% \item \texttt{ker\_data\_malloc} is used to specify the creation of the
%% message data buffer
%% %
%% \item \texttt{ker\_malloc} is used to allocate the message header (that
%% includes a reference to the data buffer)
%% %
%% \end{itemize}
%% %
%% The analysis uses \texttt{sos\_state\_malloc} rather than \texttt{ker\_malloc}
%% so that the node state can be differentiated as being persistent (note that
%% there is no other persistent data in SOS).
%% %
%% One limitation of the current analysis is the field insensitivity that leads
%% to access to fields of structs triggering false positives (none found in SOS
%% code base but one found in unit testing). 

%% A CIL module needs to be written that verifies that when a module gains
%% ownership of a buffer a persistent alias to the buffer is created, the buffer
%% is released to another module, or the buffer is deallocated.


%% A CIL module needs to be written that verifies that the attributes (or lack of
%% attributes) assigned to formal parameters are not violated by function
%% implementation.
%% %
%% From this module, a more advanced module should be written to automate the
%% generation and annotation of functions within SOS.


%% The end ownership module will check only a single module at a time.
%% %
%% This is made possible since entry and exit points from the module have well
%% defined ownership semantics.
%% %
%% Functions in other modules and the kernel that are called by a checked module
%% will be verified on demand.


%% The ability to conditionally claim ownership of data may make this a bit more
%% exciting.


%% %%%%%%%%%%%%%%%%%%%%%%%%%%%%%%%%%%%%%%%%
%% \subsection{Interprocedural Analysis}

%% We extend the basic dataflow algorithm to procedures using resource management
%% annotations at function boundaries.
%% Specifically, we annotate formal arguments and return parameters of functions
%% to describe parameters the function allocates or frees. 
%% With these annotations, the algorithm can substitute function calls by 
%% allocation or free calls, and check each function in isolation.


%% \begin{description}
%% %%%%%%%%%%%%%%%%%%%%
%% \item[release v]
        
%% Pre: v must be a reference to a memory location.

%% Post: v must be treated as dead until end of function.  Thus, it may not
%% appear in any other term T.  (This can be improved by allowing freed data to
%% be reallocated.)

    
%% %%%%%%%%%%%%%%%%%%%%
%% \item[claim v]

%% Pre: v must be an "unused" or "available" variable.

%% Post: v must be Stored or Freed exactly once on each path from claim point to
%% end of function.  (This can be improved by allowing allocated data to be
%% reallocated after it has been Freed / Stored).


%% %%%%%%%%%%%%%%%%%%%%
%% \item[borrow v]

%% Pre: None.

%% Post: No effect on v.

%% \end{description}


%% %%%%%%%%%%%%%%%%%%%%%%%%%%%%%%%%%%%%%%%%
%% \subsection{Callee}

%% \begin{description}


%% %%%%%%%%%%%%%%%%%%%%
%% \item[release v]

%% Pre: None.

%% Post: Function must Store or Free v exactly once on each path to return.


%% %%%%%%%%%%%%%%%%%%%%
%% \item[claim v]

%% Pre: None.

%% Post: Function must have Released (or Malloced) data into v along each path to
%% return.


%% %%%%%%%%%%%%%%%%%%%%
%% \item[borrow v]

%% Pre: None.

%% Post: Function may only Read / Set v.

%% \end{description}


