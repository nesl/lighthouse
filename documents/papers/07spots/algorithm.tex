\section{Analysis Specification and Implementation}
\label{sec:alg}

Motivated by the observation that a simple ownership protocol can be
used to reason about proper resource usage common to sensor networks,
we developed Lighthouse to look for violations of this protocol within
SOS code.  Lighthouse checks two key properties relating to resources:
No resource should be leaked, and there should be no dangling
pointers.  More precisely:

\smallskip\noindent {\bf No Leaks.} This first property defines the
absence of resource leaks by enforcing that along every program
execution path, every allocation is necessarily followed by either a
free or a store of a handle that refers to the same resource that was
just allocated.

\smallskip\noindent {\bf No Dangling Pointers.} The second property
states that once a resource has been freed, it is never accessed
again.

We now explain how these two properties are checked using the
Lighthouse tool, and mention some of the limitations of our implementation.

%
% NOTE: This was the formal LTL specification that is not very popular
% with the reviewers.
%
% \subsection{Specification}
% 
% We now formalize the logical specification of our ownership-based
% resource management protocol.  For simplicity, we specify the protocol
% on an idealized imperative language that includes the fundamental
% operations for resource management.  The language's atomic statements
% are as follows, where {\tt x} and {\tt y} range over {\em resource
% handle names}:
% %
% \[
% %
% y := x\mid \mathrm{alloc}\ x \mid \mathrm{free} \ x \mid \mathrm{get}\
% x \mid \mathrm{store}\ x 
% %
% \]
% %
% Programs are constructed by sequencing statements of the above form,
% and we assume the presence of standard control-flow constructs
% including conditionals and loops.
% 
% Our language's statements abstractly capture resource-related program
% operations and have the following informal semantics.  An assignment
% $y :=  x$ assigns the resource referenced by handle $x$ to the handle
% $y$.  The operation $\mathrm{alloc}\ x$ allocates a new resource and
% returns a handle $x$ to the resource.  The operation $\mathrm{free}\
% x$ frees the resource referenced by the handle $x$.  In addition, we
% assume the presence of a persistent store that holds a single
% resource.  The operation $\mathrm{store}\ x$ puts the resource
% referenced by handle $x$ into the store, and the operation
% $\mathrm{get}\ x$ assigns the resource contained in the persistent
% store to $x$.
% 
% Given this core language, we formally specify rules for proper
% resource usage as invariants on each dynamic program execution.  A
% natural formalism for these invariants is {\em linear temporal logic}
% (LTL)~\cite{Emerson90,MannaPnueli92vol1}, which we briefly review.  An
% LTL formula is constructed from atomic predicates using boolean
% operations and the {\em temporal operators} $\Box$, $\Diamond$, and
% $\Next$.  An LTL formula is evaluated at a given execution state with
% respect to a fixed execution path through the program.  The formula
% $\Box f$ holds at state $s$ if the formula $f$ holds on $s$ and each
% subsequent state in the path.  The formula $\Diamond f$ holds at state
% $s$ if the formula $f$ holds on $s$ or some subsequent state in the
% path.  The formula $\Next f$ holds at state $s$ if the formula $f$
% holds on the successor state of $s$ along the path.
% 
% 
% For a handle $x$, we write $\access(x)$ for any operation that
% syntactically accesses the resource referenced by $x$, that is, one of
% $\mathrm{free}\ x$, $\mathrm{store}\ x$, or an assignment with $x$ on
% the right-hand side.  We also introduce a relation $\alias(x,y)$ on
% pairs of handles: $\alias(x,y)$ holds at an execution state iff
% handles $x$ and $y$ refer to the same resource in that state.  We now
% define two key correctness invariants for proper resource management. 
% 
% 
% %%%%%%%%%%%%%%%%%%%%%%%%%%%%%%%%%%%%%%%%%%%%%%%%%%%%%%%%%%%%
% \smallskip\noindent
% {\bf No Leaks.} The first property formalizes the absence of resource
% leaks by enforcing that along every program execution path, every
% allocation $\mathrm{alloc}\ x$ is necessarily followed by either a
% free or a store of a handle that refers to the same resource as $x$:
% %
% \[
% %
% \Box \left( (\mathrm{alloc}\ x) \rightarrow
% %
%     \Diamond \left( (\mathrm{free}\ y \vee \mathrm{store}\ y) \wedge
% \alias(x,y) \right) \right)
% %
% \]
% 
% %%%%%%%%%%%%%%%%%%%%%%%%%%%%%%%%%%%%%%%%
% \smallskip\noindent
% {\bf No Dangling Pointers.} The second property states that once a
% resource has been freed, it is never accessed again:
% %
% \[
% %
% \Box ( \mathrm{free}\ x \rightarrow \Next (\Box (\lnot (\access(y)
% \wedge \alias(x,y))))) 
% %
% \]


\subsection{Implementation}

Our tool checks programs at compile time for violations of the above
two properties.  The tool performs a static dataflow analysis for each
property on a program's control-flow graph (CFG), which statically
represents all possible execution paths.  

For the SOS instantiation the $\mathrm{alloc}$ and $\mathrm{free}$
operations are respectively represented by {\tt ker\_malloc} and {\tt
ker\_free}.  To identify a module's persistent state in SOS, we
provide a new attribute {\tt lh\_state} for formal parameters.  For
example, the {\tt state} argument to a module's event handler would be
annotated with this attribute.  This is required since SOS modules are
passed in a reference to their persistent state, rather than directly accessing a
global variable.

Our dataflow analyses are implemented in the CIL front end for
C~\cite{CIL}, which parses C code into a simple intermediate format
and provides a framework for performing analyses on the intermediate
code.  Each analysis is {\em modular}, considering each module
independently and analyzing the CFG for each procedure within the
module in isolation.  Care must be taken that our static checker
conservatively approximate the dynamic conditions that must be
satisfied.  Two notable issues are the treatment of pointer aliasing
and of procedure calls, which we discuss in turn.

\smallskip\noindent
{\bf Pointer Aliasing.}
First, the invariants described above depend on dynamic alias
information, which cannot be exactly computed at compile time.
Instead, two standard approximations are {\em must-alias} analysis and
{\em may-alias} analysis.  A must-alias analysis underapproximates the
dynamic alias relations:  if a must-alias analysis determines that $x$
and $y$ alias at a particular program point, then $\alias(x,y)$
definitely holds at any run-time execution state corresponding to that
program point.  A may-alias analysis overapproximates the dynamic
alias relations:  if a may-alias analysis determines that $x$ and $y$
{\em cannot} alias at a particular program point, then $\alias(x,y)$
definitely does not hold at any run-time execution state corresponding
to that program point.

Our checker requires both kinds of static alias approximations.  In
the first property described above, alias information is used to
ensure that something definitely happens, namely that an allocated
resource is eventually freed.  Therefore, in this case we approximate
the true alias information with must-alias information.  In the second
property described above, alias information is used to ensure that
something definitely does not happen, namely an access to a freed
resource.  Therefore, in this case we approximate the true alias
information with may-alias information.

Our implementation currently uses a simple flow-sensitive must-alias
analysis.  For the may-alias analysis, we use a fast flow-insensitive
alias analysis provided by the CIL framework.  It should be noted that 
static alias analysis
can be imprecise in the presence of complex pointer
manipulations and pointer structures.  For example, CIL's may-alias
analysis does not distinguish among the fields of a structure, instead
considering them to always potentially alias one another.  This does not
effect the correctness of our tool, but rather causes an overly 
conservative analysis in some situations.

\smallskip\noindent
{\bf Interprocedural Analysis.}
Our analysis is {\em modular}:  each function in a module or component
is analyzed in isolation.  To make checking of a function body precise
in the presence of calls to other functions, we employ {\em ownership
attributes} for function headers that capture the memory-related
behavior of a called function.  We add two attributes to the C code:
{\tt lh\_claim} and {\tt lh\_release}.  A formal parameter or return
value that has the {\tt lh\_claim} attribute indicates that the caller
must take ownership of the associated memory after a call.  This
annotation, for example, would be used to annotate a function that
wraps a call to {\tt ker\_malloc} within SOS, allowing that function's
callers to be properly checked without access to the function's
implementation.  Similarly, an {\tt lh\_release} attribute on a formal
parameter indicates that ownership of the parameter is transferred
from the caller of the function to the callee.  If a parameter does
not have an ownership attribute, memory ownership is unchanged.  Our
tool ensures that these attributes are employed wherever necessary,
when checking the implementation of each function.  In practice, we
have found that a small set of annotations is sufficient for precise
analysis.

%Second, our simple core language used to specify the resource
%invariants does not contain procedure calls, which must be properly
%handled in our implementation.  As mentioned above, our analyses are
%{\em intraprocedural}, meaning that each procedure is checked in
%isolation.  To make such checking both correct and precise, we rely on
%the ownership attributes {\tt lh\_claim} and {\tt lh\_release}, as
%described in the previous section.

Given such attributes, the proper handling of procedure calls becomes
straightforward.  A procedure call statement is treated logically by
the checker as an assignment from actuals to formals, followed by an
assignment from the return value of the call to the left-hand-side
variable (if any).  A formal parameter annotated with the {\tt
lh\_claim} attribute is treated as an allocation site, just as is {\tt
ker\_malloc}.  A formal parameter annotated with the {\tt lh\_release}
attribute is treated as a disposal site, just as is {\tt ker\_free}.

The ability to perform checking modularly allows application writers
to obtain early feedback about the correctness of their resource
management, without requiring access to the rest of the system.  This
is particularly important in a system like SOS, in which modules can
be linked and unlinked dynamically.  In such a setting, the ``rest''
of the system is a moving target, so it is not really possible to
consider whole-program analysis.



\subsection{Limitations}

As we demonstrate in the next section, our checker is useful for
detecting violations of the ownership protocol on real sensor-network
code.  However, the checker is not guaranteed to find all such
violations.  In other words, the checker can be used for finding
memory errors but not for guaranteeing the absence of all memory
errors.  We discuss three limitations of the checker in this regard.

First, the checker does not precisely handle all of the unsafe
features of the C programming language.  For example, pointer
arithmetic is not statically analyzed.  Instead, an expression of the
form $p+i$, where $p$ is a pointer and $i$ is an integer, is simply
treated as if it refers to the same block of memory as $p$.  If $p+i$
in fact overflows to another block of memory at run time, the
checker's assumption can cause it to miss errors.  These kinds of
limitations are standard for C-based program analyses.

Second, there is a design choice about how to treat resources that a
handler does not explicitly acquire through an interface that is
annotated with {\tt lh\_claim} and {\tt lh\_release}.  Technically the
ownership protocol disallows these resources from being accessed at
all, since the receiving module is not the owner.  However, enforcing
this requirement can cause spurious errors for patterns of resource
sharing that are in fact safe, for example when a module temporarily
``borrows'' data from another module within a bounded scope.
Therefore, our checker does not currently enforce the requirement that
a module only access a resource that it owns, reducing the number of
false positives but also potentially missing errors.  The checker
still ensures that a handler does not access a resource that has
earlier been freed or released.

Finally, our modular checker does not consider the overall order in
which messages will be received.  However, there may be
application-level protocols that determine how events are generated in
the system, and these protocols can affect the correctness of
ownership tracking.
%
For example, imagine a module that responds to three events: $\create$
causes the module to allocate a block $b$ and store it into the
module's persistent store, $\access$ causes the module to access block
$b$ via the  store, and $\delete$ causes the module to deallocate
block $b$ via the store.  The module properly manages block $b$ as
long as the temporal sequence of events follows the regular expression
$\alloc$ $\access^*$ $\delete$.  This ordering ensures that the system
never accesses the block before it is allocated and always eventually
deletes the block.

Our tool currently only tracks data locally within each event handler.
When performing this tracking, the tool assumes on entry that all data
in the persistent store is owned by the module.  The tool similarly
considers a resource to be properly released when it is placed in the
persistent store.  These assumptions can miss errors, for example if
an $\access$ event ever occurs before the $\alloc$ event.  However,
our assumptions provide a practical point in the design space that
allows each event handler to be usefully checked for local violations
of the ownership protocol.
%
Handling these inter-module relationships and application-level
protocols is something we are interested to include in our future
work~\cite{AlurPOPL05,HJM05}.
%
%It will be interesting to extend our system with additional interface
%annotations about application-level protocols in the
%future~\cite{AlurPOPL05,HJM05}.


