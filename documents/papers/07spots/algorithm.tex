\section{The Lighthouse Tool}
\label{sec:alg}

We have instantiated our approach to static checking of dynamic
resource management in Lighthouse, a tool for ensuring proper
ownership of dynamic memory in SOS programs.  The tool employs a suite
of dataflow analyses to enforce the three rules described in
Section~\ref{subsec:owner} on each function.  
This section overviews the usage and implementation of Lighthouse.

% Motivated by the observation that a simple ownership protocol can be
% used to reason about proper resource usage common to sensor networks,
% we developed Lighthouse to look for violations of this protocol within
% SOS source code.  
% %
% Lighthouse checks two key ownership properties properties
% relating to resources: No resource should be leaked, and there should
% be no dangling pointers.  
% %
% More precisely:

% \smallskip\noindent {\bf No Leaks.} This first property defines the
% absence of resource leaks by enforcing that along every program
% execution path, every allocation is necessarily followed by either a
% free or a store of a handle that refers to the same resource that was
% just allocated.

% \smallskip\noindent {\bf No Dangling Pointers.} The second property
% states that once a resource has been freed, it is never accessed
% again.


\subsection{Ownership Annotations}

Lighthouse depends on programmer-specified 
{\em ownership annotations} to specify when ownership of a block of
memory should transfer from one component to another.
%
These annotations provide a ``spec'' for each function,
indicating its preconditions and postconditions with respect to memory
ownership.  This spec then enables precise modular checking of functions.  Each
function body is checked to ensure the postconditions on exit, under the
assumption that the preconditions hold on entry.  The function is also
checked to obey the three rules constituting the ownership discipline,
defined earlier.  Each
caller is separately checked to ensure the callee's preconditions
before the call, and
it may assume the callee's postconditions after the call.

Two ownership annotations are used to describe changes of memory ownership
resulting from function calls: {\tt lh\_claim} and {\tt lh\_release}.
%
The {\tt lh\_claim} annotation states that the caller of the function
will take ownership of the memory pointed to by the annotated formal
parameter or annotated return value, and therefore that the function
body must release it.
For example, the SOS kernel's
\code{ker\_malloc} function has the {\tt lh\_claim} annotation on its
return value, to indicate that it returns dynamically allocated memory
that the caller will own.
%
Conversely, the {\tt lh\_release} annotation states that the caller of
the function will release ownership of 
the memory pointed to by the annotated formal parameter,
and therefore that the function body must take ownership of this memory.
For example, the formal parameter of the SOS kernel's \code{ker\_free}
function has the {\tt lh\_release} annotation.
%

Our ownership discipline defined earlier relies 
on the notion of a {\em persistent
store}:  each component is assumed to own the data in its persistent
store and can properly ``release'' owned data into this store.
%
All global and static variables declared with a component 
are assumed by the analysis to be part
of the component's persistent store.
%
A third and final annotation, {\tt lh\_store}, is used to denote formal
parameters of functions that can also be considered part of the
persistent store.
%
An example usage of {\tt lh\_store} is to annotate 
the \code{state} parameter of a
module's message handler in SOS.
%
As described earlier, 
this parameter points to the persistent store allocated and
maintained by the SOS kernel for a module.

% Given such annotations, the proper handling of procedure calls becomes
% straightforward.  A procedure call statement is treated logically by
% the checker as an assignment from actuals to formals, followed by an
% assignment from the return value of the call to the left-hand-side
% variable (if any).  A formal parameter annotated with the {\tt
% lh\_claim} annotation is treated as an allocation site, just as is {\tt
% ker\_malloc}.  A formal parameter annotated with the {\tt lh\_release}
% annotation is treated as a disposal site, just as is {\tt ker\_free}.

Annotations typically reside in an external configuration file listing
a function name and the corresponding annotations for the return value
and formal parameters.
%
Only functions that require annotations need to be included within
this configuration file.
%
Annotations can also be included directly in the source code as GCC
attributes decorating the prototype of a function.
%
% Lighthouse ensures that each parameter with an annotation properly
% follows the ownership protocol within the body of the function.
%
In practice we have found that a small set of annotations, \numannote for the
complete evaluation of SOS, is sufficient for precise
analysis. 

The ability to perform checking modularly allows application writers
to obtain early feedback about the correctness of their resource
management, without requiring access to the rest of the system.  This
is particularly important in a system like SOS, in which modules can
be linked and unlinked dynamically.  In such a setting, the ``rest''
of the system is a moving target, so it is not really possible to
consider an approach based on whole-program analysis.


\subsection{Implementation Overview}

%
Lighthouse is implemented in the CIL front end for C~\cite{CIL},
which parses C code into a simple intermediate format and provides a
framework for performing analyses on the intermediate code. 
%
Lighthouse takes as input a preprocessed C file and prints out error
messages similar to those produced by a C compiler when suspect
code is identified.
%
The analysis does not modify the preprocessed code and can be
trivially called from a makefile between the preprocessing and code
generation stages of compilation.


The Lighthouse analysis traverses each function's control-flow graph
(CFG) in isolation, which conservatively represents all possible execution
paths through the function.  During this traversal, Lighthouse
triggers two major dataflow analyses to detect potential errors:

\begin{itemize}
\item Whenever a node in the graph is
encountered that allocates or takes ownership of a block of memory,
Lighthouse invokes a dataflow analysis to ensure that every path from
this node to the function exit frees, stores, or releases ownership
of the memory exactly once.  If this property is not satisfied,
Lighthouse reports a possible memory leak.

\item Whenever a node in the graph is encountered that frees or
  releases ownership of a block of memory, Lighthouse invokes a
  dataflow analysis to ensure that no path from this node to the
  function exit accesses the memory.  If this property is not
  satisfied, Lighthouse reports a possible dangling pointer error.
\end{itemize}

These two analyses also serve to check that a function meets its
spec.  For the purpose of the Lighthouse traversal over a function's CFG,
the function's entry node is considered
an allocation point for all parameters with the {\tt lh\_release}
annotation, and the function's exit node is considered a release point
for all parameters and return values with the {\tt lh\_claim}
annotation.  


\subsection{Pointer Aliasing}

The analysis described above requires knowledge of the memory
pointed to by a function's pointers.  As usual, this is
statically approximated by an {\em alias analysis}, which determines
whether two different pointers store the same memory location at a
given program point.  
 Two standard approximations to the true dynamic alias information 
are {\em must-alias} analysis and
{\em may-alias} analysis.  A must-alias analysis underapproximates the
dynamic alias relations:  if a must-alias analysis determines that $x$
and $y$ alias at a particular program point, then they
definitely alias at that point in 
any program execution.  A may-alias analysis overapproximates the dynamic
alias relations:  if a may-alias analysis determines that $x$ and $y$
{\em cannot} alias at a particular program point, then they
definitely do not alias at that point in any program execution.

Lighthouse requires both kinds of static alias approximations.  In
the first dataflow analysis described above, alias information is used to
ensure that something definitely happens, namely that an allocated
resource is eventually released.  Therefore, in this case we approximate
the true alias information with must-alias information.  In the second
dataflow analysis described above, alias information is used to ensure that
something definitely does not happen, namely an access to a released
resource.  Therefore, in this case we approximate the true alias
information with may-alias information.

We have built a simple flow-sensitive must-alias analysis for use by
Lighthouse.  
For the may-alias analysis, we use a fast flow-insensitive
analysis provided by the CIL framework.  
Obtaining precise
alias information at compile time is notoriously difficult, and this
limitation is the principal cause of false positives for our analysis.
For example, CIL's may-alias
analysis does not distinguish among the fields of a structure, instead
considering them to always potentially alias one another.  Both alias
analyses can be imprecise in the presence of linked data structures.


\subsection{Limitations}

As we demonstrate in the next section, our checker is useful for
detecting violations of the ownership protocol on real sensor network
code.  However, the checker is not guaranteed to find all such
violations.  In other words, the checker can be used for finding
memory errors but not for guaranteeing the absence of all memory
errors.  The checker's false negatives come from three sources.

First, the checker does not precisely handle all of the unsafe
features of the C programming language.  For example, pointer
arithmetic is not statically analyzed.  Instead, an expression of the
form $p+i$, where $p$ is a pointer and $i$ is an integer, is simply
treated as if it refers to the same block of memory as $p$.  If $p+i$
in fact overflows to another block of memory at run time, the
checker's assumption can cause it to miss errors.  These kinds of
{\em memory safety} assumptions are standard for C-based program analyses.

Second, there is a design choice about how to treat resource
parameters that are not
annotated with {\tt lh\_release}.  Technically the
ownership protocol disallows these resources from being accessed at
all, since the receiving module is not the owner.  
%
Lighthouse does not currently enforce the requirement that a module
only access a resource that it owns.
% 
This design decision provides a seamless adoption path for Lighthouse
on existing systems.  Programmers can incrementally add
annotations to refine the quality of the analysis without
suffering from excessive numbers of false positives.
%
This decision also provides a mechanism for allowing patterns of resource
sharing that are in fact safe but not supported by the ownership
discipline, for example when a module temporarily ``borrows'' data from
another module within a bounded scope.
%
The cost of this flexibility is the potential to miss real resource
management errors.
%
However, the checker still ensures that a function does not access a
resource that has earlier been released by the function, thereby
detecting memory leaks.

Third, our modular checker does not consider the overall order in
which messages will be received by a component.  
%
However, event orderings
can affect the correctness of ownership tracking when a resource is
accessed via the persistent store.
%
For example, consider a module that responds to three events: $\create$
causes the module to allocate a block $b$ and store it into the
module's persistent store, $\access$ causes the module to access block
$b$ via the store, and $\delete$ causes the module to deallocate
block $b$ in the store.  
%
The module avoids dangling accesses to $b$ and leaks of $b$ 
as long as the temporal sequence
of events follows the regular expression $($ $\alloc$ $\access^*$
$\delete$ $)^*$.  
%
%

Lighthouse currently only tracks data locally within each
event handler.  
%
When performing this tracking, the tool assumes on entry that all data
in the persistent store is owned by the module.
%
The tool similarly considers a resource to be properly released when
it is placed in the persistent store.  
%
These assumptions allow each of the three event handlers in our
example to pass all checks.  Nonetheless, dynamic memory errors can still happen, 
for example if an $\access$ event
ever occurs before the $\alloc$ event.  
%
Our assumptions provide a practical point in the design space
that allows each event handler to be usefully checked for local
violations of the ownership protocol.  
%
Handling inter-module relationships and application-level
protocols is something we plan to pursue in the future, leveraging
recent work on {\em interface synthesis} 
for software components~\cite{AlurPOPL05,HJM05}.

