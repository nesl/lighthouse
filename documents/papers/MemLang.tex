\documentclass{article}

\begin{document}

%%%%%%%%%%%%%%%%%%%%%%%%%%%%%%%%%%%%%%%%%%%%%%%%%%%%%%%%%%%%
\section{Overview}

This is a revised version of the properties that we want the checker to
verify.  It is based on the discussion from our meeting on March 7, 2006.  As
always, please update any problems that you see with this.  I am leaving the
high level description of evaluating code at the end of this document.  While
it is deprecated by the formal specification (and may contradict it in some
locations), it also provides a nice description of what the CIL checkers were
targeted towards.


%%%%%%%%%%%%%%%%%%%%%%%%%%%%%%%%%%%%%%%%%%%%%%%%%%%%%%%%%%%%
\section{Terms}

\begin{eqnarray*}
t & ::= & \mathrm{op} \ x \\
%
& & \\
%
\mathrm{op} & ::= & \mathrm{malloc} \ x \ | \\
        & & \mathrm{free} \ x \ | \\ 
        & & \mathrm{get}_{y} \ x \ | \\
        & & \mathrm{store}_{y} \ x \\
%
& & \\
%
\mathrm{prog} & ::= & t_{1}; \ \ldots; \ t_{n}
\end{eqnarray*}


%%%%%%%%%%%%%%%%%%%%%%%%%%%%%%%%%%%%%%%%%%%%%%%%%%%%%%%%%%%%
\section{Evaluation Rules}

%%%%%%%%%%%%%%%%%%%%%%%%%%%%%%%%%%%%%%%%
\subsection{No Memory Leaks}

\begin{displaymath}
%
\forall \ x, \ t_{i} = \mathrm{malloc} \ x \ 
%
\Rightarrow \
%
\exists \ j > i \ 
%
\mathrm{st.} \
%
t_{j} = \mathrm{free} \ [\mathrm{must \ alias} \ x] \ 
%
\vee \ t_{j} = \mathrm{store}_{y} \ [\mathrm{must \ alias} \ x] 
\end{displaymath}

This property is verified by the CIL data flow module \texttt{ownFlow.ml} called
by \texttt{memory.ml}.  This data flow is called from every allocation point,
and verifies that the resulting value is released or inserted into a store.

%%%%%%%%%%%%%%%%%%%%%%%%%%%%%%%%%%%%%%%%
\subsection{No Dangling Pointers}

\begin{displaymath}
%
\forall i, \ t_{i} = \mathrm{free} \ x \ 
%
\Rightarrow \
%
\forall \ j > i, \ t_{j} \neq \mathrm{op} \ [\mathrm{may \ alias} \ x] 
%
\end{displaymath}

This property is verified by the CIL data flow module \texttt{deadFlow.ml}
called by \texttt{memory.ml}.  This data flow is called from every release
site, and verifies that the released expression is treated as ``dead'' from
that point forward in the program.  Note that the expression may appear as an
l-value.

%%%%%%%%%%%%%%%%%%%%%%%%%%%%%%%%%%%%%%%%
\subsection{Allocation Before Use}
\label{ss_alloc_before}

\begin{displaymath}
%
\forall \ i, \ t_{i} = \mathrm{op} \ x \ 
%
\Rightarrow \
%
\exists \ j \geq i \ 
%
\mathrm{st.} \
%
t_{j} = \mathrm{malloc} \ [\mathrm{must \ alias} \ x] \ 
%
\vee \ t_{j} = \mathrm{get}_{y} \ [\mathrm{must \ alias} \ x] 
\end{displaymath}

This property is not verified by the checker.

%%%%%%%%%%%%%%%%%%%%%%%%%%%%%%%%%%%%%%%%
\subsection{Function Calls and Attributes}

The formal description described above assumes a single linear program.  SOS
modules need not be linear and make use of function calls.  Modular checking
can be accomplished by annotating function calls to describe the parameters
the the function frees or allocates.  The checker assumes that these
annotations are correct and treats annotated function calls as a collection of
allocation and free sites.

Further, the checker can verify that the annotations provided on a function
are correct.  From the callees perspective, annotations can be thought of as a
collection of allocation points immediately before the body of a function and
of free points immediately after the body of a function.


The framework for these attributes includes the use of the following CIL
modules:

\begin{description}
%
\item[\texttt{addAnnotations.ml}]  Inserts annotations into unannotated code.
This keeps the SOS code base clean and provides a convenient way to manage the
annotations.
%
\item[\texttt{fillFlow.ml}] Used to verify that formal parameters that must
release data to the caller are allocated or ``filled'' by the function.  Note
that this data flow may be leveraged to verify the property described
in section~\ref{ss_alloc_before}.
\end{description}



%%%%%%%%%%%%%%%%%%%%%%%%%%%%%%%%%%%%%%%%%%%%%%%%%%%%%%%%%%%%
\section{Evaluation In Light of Attributes}

The framework described above can be thought of in terms of pre and post
conditions for callers and callees of functions.  This form may help provide
an intuitive feeling for what the checker is doing.  Note that this section
has been deprecated by the formal description provided above.


%%%%%%%%%%%%%%%%%%%%%%%%%%%%%%%%%%%%%%%%
\subsection{Caller}

\begin{description}


%%%%%%%%%%%%%%%%%%%%
\item[release v]
        
Pre: v must be a reference to a memory location.

Post: v must be treated as dead until end of function.  Thus, it may not
appear in any other term T.  (This can be improved by allowing freed data to
be reallocated.)

    
%%%%%%%%%%%%%%%%%%%%
\item[claim v]

Pre: v must be an "unused" or "available" variable.

Post: v must be Stored or Freed exactly once on each path from claim point to
end of function.  (This can be improved by allowing allocated data to be
reallocated after it has been Freed / Stored).


%%%%%%%%%%%%%%%%%%%%
\item[borrow v]

Pre: None.

Post: No effect on v.

\end{description}


%%%%%%%%%%%%%%%%%%%%%%%%%%%%%%%%%%%%%%%%
\subsection{Callee}

\begin{description}


%%%%%%%%%%%%%%%%%%%%
\item[release v]

Pre: None.

Post: Function must Store or Free v exactly once on each path to return.


%%%%%%%%%%%%%%%%%%%%
\item[claim v]

Pre: None.

Post: Function must have Released (or Malloced) data into v along each path to
return.


%%%%%%%%%%%%%%%%%%%%
\item[borrow v]

Pre: None.

Post: Function may only Read / Set v.

\end{description}


\end{document}

