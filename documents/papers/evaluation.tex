\section{Evaluation}
\label{sec:eval}


\subsection{Quantifying Dynamic Memory Usage in SOS}

%% \begin{table}
%% \begin{small}
%% \centering 
%% \begin{tabular}{| l | r | r |} 
    
%%     \hline

%%     \textbf{Function} & \textbf{Total uses} & \textbf{Module uses} \\
%% % & \textbf{of 46371 SLOC} & \textbf{of 12990 SLOC} \\

%%     \hline
%%     \multicolumn{3}{|c|}{\textbf{Primary operations to allocate memory}} \\
%%     \hline

%%     ker\_malloc & 135 & 68 \\
%%     \hline
%%     ker\_msg\_take\_data & 25 & 23 \\
%%     \hline
%%     msg\_create & 15 & 1 \\

%%     \hline
%%     \multicolumn{3}{|c|}{\textbf{Primary operations to release memory}} \\
%%     \hline

%%     ker\_free & 124 & 46 \\
%%     \hline
%%     SOS\_MSG\_RELEASE & 116 & 66 \\

%%     \hline
%%     \multicolumn{3}{|c|}{\textbf{Summary of other memory manipulations}} \\
%%     \hline
    
%%     Other memory & 39 & 7 \\
%%     \hline

%% \end{tabular} 
%% %
%% \end{small}
%% \centering
%% %
%% \caption{Memory usage in SOS}
%% %
%% \label{tab:alloc}
%% %
%% \end{table}

First we performed a study to understand the prevalence of
dynamic memory operations in the SOS source.
The SOS API includes a number of functions that manipulate memory.  These
functions include {\tt ker\_malloc} and {\tt ker\_free} to allocate and
free data, {\tt ker\_msg\_take\_data} and {\tt SOS\_MSG\_RELEASE} used to
transfer data ownership, and a handful of functions that employ these
primitives to 
allocate specific kinds of data structures.  Examining the SOS CVS head from April
2006 reveals that nearly one memory operation appears per
100 source lines of code (SLOC).  Of the 46371 SLOC examined, 12990 are from
modules written by end users to build applications, rather than from
the implementation of SOS itself and associated drivers.  The frequency of
memory operations in end-user modules is even higher than the overall average,
with one such operation for every 60 SLOC.  
%% Table~\ref{tab:alloc} summarizes the use of dynamic memory.
These findings indicate that memory management is an important part of
sensor-network programming in SOS.
%% Further, anecdotal evidence from the developers' mailing list suggests that
%% memory leaks are an important class of bugs.

\subsection{Validating SOS End-User Modules}

%% Having identified that dynamic memory management is frequent within SOS,
Next we ran our tool on all SOS end-user modules from the
April head of the SOS CVS repository.  
The goal of this experiment was to demonstrate that the tool is
practical for use as part of the standard development cycle
and to demonstrate its ability to locate errors in real sensor-network
software.

%
%% Source code for this evaluation was obtained from the April head of the
%% SOS CVS repository.  
%% The analysis tool takes as input unmodified C code that
%% has passed through the C preprocessor.  
%% We obtained the preprocessed code by 
%% adding the following target to the make system provided in SOS:
%% %
%% \begin{footnotesize}
%% \begin{verbatim}
%% %.i : %.c
%%     $(CC) -E $(CFLAGS) $(INCDIR) $< -o $@
%% \end{verbatim}
%% \end{footnotesize}
%% %

We integrated our checker in the normal SOS build process by
adding a target to the Makefile that invokes the checker on the given code
during compilation.
The average time to check a file is less than a quarter of a second.
Output from the analysis takes the form of warnings similar to those generated
during other stages of compilation.
%
The analysis generates three types of warnings:
%
\begin{description}
%
\item[Dangling pointer:] Source code is accessing data that was
released or freed at an earlier point.
%
\item[Access to dead data freed in a loop:] Special case of the prior warning.
Limitations of CIL's may-alias analysis result in
false positives when data is freed in a loop.  Separating this case out alerts
the user to a probable false positive.  Users may also set a flag
to suppress these warnings.
%
\item[Memory leak:] Source code neither stores, releases, nor frees
data that was taken or allocated.
%
%% \item[Data stored twice:] Source code appears to store data more than once.
%% In rare instances this can be a sign of an error in a program.  However, most
%% of these warnings result from artifacts of the alias analysis and can be
%% treated by the user as a probable false positive.  End users may want to
%% disable these warnings.  They are provided in this analysis for completeness.
%
\end{description}
%

\begin{table}
\caption{Warnings in SOS user modules}
%
\label{tab:module}
\centering 
\begin{tabular}{| l | r |}
%%     \hline 
%%     Failed to compile & 8 \\
    \hline 
    Actual memory leaks & 16 \\
    \hline
    Missing annotations & 153 \\
    \hline 
    Free within a loop & 66 \\
    \hline 
%%     Double take & 41 \\
%%     \hline 
    False positives & 72 \\
    \hline 
\end{tabular} 
%
\end{table}

%% \smallskip\noindent{\bf SOS CVS.}
We applied the checker to every historic version of
each user module included in the SOS CVS repository.  Of the 203 historic versions
of the 37 modules available, 77 versions resulted in
warnings from the checker, eight versions caused 
 the CIL parser to crash before checking could occur, and the
 remaining 118 versions passed the checker with no warnings.
%% A small set of functions that properly compile cause cause the base CIL~\ref{XXX} file parser to crash.  
We have not yet had the opportunity to resolve the problem with CIL's
parser.

A total of 307 warnings were generated by the checker during this experiment.
Each warning was examined by hand and classified as an
actual error or a false positive; the results are presented in
Table~\ref{tab:module}.  Sixteen of the warnings turned out
to be real memory leaks in the code, resulting from four distinct
errors 
that remained in place across multiple CVS versions.  An example
memory leak from the code is described at the end of this section.

The 291 remaining warnings were further classified to better understand the
sources of imprecision in 
the current implementation of the checker.  
%

\smallskip
\noindent{\bf Missing annotations.}
%% Instead of annotating the SOS codebase
%% with ownership attributes
%% up front, we iteratively added annotations to function parameters
%% based on warnings from the checker.
%% The analysis framework that we implemented dynamically adds attributes the formal parameters
%% of functions.  The configuration for this is contained in a simple format that
%% lists the function name and parameter attributes for functions that require
%% annotation.  We were able to quickly generate a configuration file containing
%% 20 function annotations specifically for the SOS CVS head by iteratively
%% analyzing a file and looking at functions causing the analysis to generate
%% warnings and using this configuration for all of our analysis.  
%% Stored data must be placed in a persistent location.  Since SOS modules do not allow
%% global variables, event handlers in SOS are passed a formal parameter
%% pointing to the module's state. This is the one valid store available to the
%% module.  For this analysis we annotate the state parameter of each module's
%% handler function as a store.
We ran the checker with 20 ownership attributes on functions from the
core SOS API, as well as an {\tt sos\_state} attribute for the {\tt
  state} argument to each module's message handler.
%% Our final evaluation used a fixed set of 20 annotations to functions from the
%% core SOS API, in addition to one annotation for the primary event handler
%% function of each module examined.
While testing the tool on historic versions, 153 warnings were due to missing
annotations.
These warnings consist of 60 due to changes in the behavior of
SOS API functions
over time, 49 due to
(currently deprecated) functions in the core SOS API that required
annotations and are used in historic versions of 
modules,
and 44 warnings from the need for additional {\tt sos\_state}
attributes (for example, when a message handler's {\tt state} argument
is passed as a parameter to a helper function).
%% To help automate the testing of the many historic versions
%% of SOS modules we did not update are configuration for theses false positives.
%% Based on our experience generating the annotation configuration for the SOS
%% head, we are confidant that the few annotations needed to remove these false
%% positives could be iteratively added as code is developed with little burden to
%% the end user.
All of
these warnings are eliminated once the appropriate annotations
are added to the code.

\smallskip
\noindent{\bf Free within a loop.}
As mentioned earlier, imprecisions in the may-alias analysis cause
warnings to be signaled when data is freed within a loop.
There were 66 such false positives across all historical versions of
the end-user modules.  As mentioned earlier, our tool includes an
option to suppress these warnings, which is enabled by default.  

\smallskip
\noindent{\bf Other false positives.}
After elimination of the above two classes of warnings (either
by annotation or by suppression), we are left with 72 false
positives, which come from three main sources.
The primary culprit is CIL's flow-insensitive field-insensitive
may-alias analysis, which is particularly imprecise
for reasoning about deep data structures.  
%% However, our framework simply calls into the alias analysis.  
%% This will allow us to easily replace
%% the current alias analysis with a more powerful framework at any time.
We plan to experiment with more sophisticated alias analyses to
eliminate many of these false positives.

A second source of false positives arises from functions that
conditionally release data in SOS.  Some functions take as input a dynamically
allocated buffer and return a status code indicating whether the data was released.
The caller of the function is responsible for checking this return value and,
based on its value, properly treating the data.
Since our analysis is not path-sensitive, these kinds of functions
result in false positives.
However, the presence of these warnings has led us to conversations with the SOS 
developers about transitioning away from this style of API.

A third source of false positives arises from application-level protocols that
ensure memory is used correctly across multiple invocations of the
module's message handler, but which cannot be validated modularly.


%% There were 153 warnings generated while configuring the tool.
%% Further, 107 warnings are from limitations of the checker known to cause false
%% positives.  These 107 false positives are listed for completeness, in the actual
%% use of the tool, these messages are suppressed.  
%% The user gets to see 88 warnings, of which 16 turn out to be real bugs.
%% The results are presented in table~\ref{tab:module} and further discussed below.

\subsection{Validating the SOS Kernel}

\begin{table}
\caption{Warnings in the SOS kernel}
%
\label{tab:kernel}
\centering 
\begin{tabular}{| l | r |}
%%     \hline 
%%     Failed to compile & 8 \\
    \hline 
    Actual memory leaks & 1 \\
    \hline 
    Actual dangling pointer errors & 1 \\
    \hline
    Missing annotations & 10 \\
    \hline 
    Free within a loop & 5 \\
    \hline 
%%     Double take & 41 \\
%%     \hline 
    False positives & 26 \\
    \hline 
\end{tabular} 
%
\end{table}
%% \begin{table}
%% \centering 
%% \begin{tabular}{| l | r |} 
%%     \hline 
%%     \multicolumn{2}{|c|}{Incomplete annotations} \\
%%     \hline
%%     Function without annotations & 10 \\
%%     \hline
%%     \multicolumn{2}{|c|}{Known limitations (hidden from user)} \\
%%     %\multicolumn{2}{|c|}{(hidden from user)} \\
%%     \hline
%%     Loop & 5 \\
%%     \hline
%%     Double take & 18 \\
%%     \hline
%%     \multicolumn{2}{|c|}{Remaining false positives} \\
%%     \hline
%%     Nested data structures & 12 \\
%%     \hline
%%     Other aliasing problems (dead access) & 14 \\
%%     \hline
%%     \multicolumn{2}{|c|}{Actual errors} \\
%%     \hline
%%     Memory leak & 1 \\
%%     \hline
%%     Invalid memory access & 1 \\
%%     \hline
%% \end{tabular} 
%% %
%% \centering
%% %
%% \caption{Warnings breakdown for core SOS}
%% %
%% \label{tab:kernel}
%% %
%% \end{table}

We ran a similar experiment on the SOS kernel.
We ran the checker on the current version of all code required to
build the core SOS kernel with a simple ``blink'' application for the
Mica2 hardware.  This configuration consists of approximately 15000 SLOC and
required analyzing 37 source files, of which 16
generated warnings.
A detailed listing of these warnings is
provided in table~\ref{tab:kernel}.
%
A total of 43 warnings are reported by the checker.

Surprisingly,
although the SOS kernel code is stable and heavily exercised, the checker found
two actual errors.
The first is a memory leak similar to the example shown in the next
subsection.
Listed here is the other error, an improper access to data that has been released to
another module.
%
\begin{footnotesize}
\begin{verbatim}
if( post_long(fst->requester, KER_FETCHER_PID,
              MSG_FETCHER_DONE, sizeof(fetcher_state_t),
              fst, SOS_MSG_RELEASE) != SOS_OK ) {
    ...
    return;
}
cam = ker_cam_lookup( fst->map.key );
...
\end{verbatim}
\end{footnotesize}
%
In this segment of code the variable {\tt fst} is released by the call to {\tt
post\_long} and subsequently accessed when the {\tt if} clause fails.
Both of
the kernel errors found by the checker
have been reported to the SOS development team.
The presence of memory errors in mature and well-tested code, written by systems
programmers, demonstrates the utility of an automatic tool for
validating memory management.

%% \subsection{Warnings Breakdown}

%% We now analyze the warnings generated in the two sets of experiments.


\subsection{A Memory Leak in SOS}
\label{ss:tale}

\begin{figure}[tp]
\begin{footnotesize}
\begin{verbatim}
mod_op = (sos_module_op_t*) ker_msg_take_data(msg);
if(mod_op == NULL) return -ENOMEM;
if(mod_op->op == MODULE_OP_INSMOD) {
    existing_module = ker_get_module(mod_op->mod_id);
    if(existing_module != NULL) {
        uint8_t ver = sos_read_header_byte(
                existing_module->header,
                offsetof(mod_header_t, version));
            if (ver < mod_op->version) {
                ker_unload_module(existing_module->pid, 
                        sos_read_header_byte(
                        existing_module->header,
                        offsetof(mod_header_t, version)));
            } else {
                return SOS_OK;
            }
        }
    ret = fetcher_request(KER_DFT_LOADER_PID,
            mod_op->mod_id,
            mod_op->version,
            entohs(mod_op->size),
            msg->saddr);
    s->pend = mod_op;
    ker_led(LED_RED_TOGGLE);
    return SOS_OK;
}
return SOS_OK;
\end{verbatim}
\end{footnotesize}
\label{fig:leak}
\caption{A memory leak in an SOS module.}
\end{figure}

The following is an example of a memory leak found by the checker.
The error was found during testing of historic versions of the file
{\tt loader.c},
which is used in SOS to dynamically load other modules onto running nodes.  
In mid-October 2005 the block of code shown in Figure~\ref{fig:leak} was
checked into CVS as part of {\tt loader.c}.
All paths through this block of code leak the {\tt
mod\_op} pointer, which the module acquires ownership of through {\tt
  ker\_msg\_take\_data}.
This code results in the following warning from the checker:

\begin{footnotesize}
\begin{verbatim}
Warning: Alloced data from instruction #line 125 "loader.c"
    mod_op = (sos_module_op_t *) ker_msg_take_data(msg);
is not stored
\end{verbatim}
\end{footnotesize}

After three additional revisions that do not significantly modify or fix the
memory leak, a fourth revision was made in mid-December 2005.  This
revision expands the functionality of {\tt loader.c} and breaks
the code up into smaller functions:

\begin{footnotesize}
\begin{verbatim}
sos_module_op_t *mod_op;
if (msg->saddr == ker_id() || s->pend) {
    return SOS_OK;
}
mod_op = (sos_module_op_t*) ker_msg_take_data(msg);
if(mod_op == NULL) return -ENOMEM;
switch(mod_op->op){
case MODULE_OP_INSMOD:
    return module_op_insmod(s,msg,mod_op);
case MODULE_OP_RMMOD:
    return module_op_rmmod(s,msg,mod_op);
}
return SOS_OK;
\end{verbatim}
\end{footnotesize}

This code again causes our checker to warn about the memory leak:

\begin{footnotesize}
\begin{verbatim}
Warning: Alloced data from instruction #line 186 "loader.c"
mod_op = (sos_module_op_t *) ker_msg_take_data(msg);
is not stored
\end{verbatim}
\end{footnotesize}

Clearly {\tt mod\_op} is still being leaked if {\tt mod\_op->op} does not
find a matching case in the {\tt switch} statement.
Further, adding the {\tt sos\_release} attribute to the third formal
parameter of the functions
{\tt module\_op\_insmod} and {\tt module\_op\_rmmod} and rerunning the
checker 
reveals that both of these
functions also leak {\tt mod\_op}!

A day later, eight weeks after the memory leaks were first
introduced, these memory leaks were found and fixed:

\begin{footnotesize}
\begin{verbatim}
sos_module_op_t *mod_op;
if (msg->saddr == ker_id() || s->pend) {
    return SOS_OK;
}
mod_op = (sos_module_op_t*) ker_msg_take_data(msg);
if(mod_op == NULL) return -ENOMEM;
switch(mod_op->op){
case MODULE_OP_INSMOD:
    return module_op_insmod(s,msg,mod_op);
case MODULE_OP_RMMOD:
    return module_op_rmmod(s,msg,mod_op);
}
ker_free(mod_op);
return SOS_OK;
\end{verbatim}
\end{footnotesize}

As shown above, a call to {\tt ker\_free} has been added before the final {\tt
  return}, in order to properly dispose of {\tt mod\_op}.
The functions {\tt module\_op\_insmod} and {\tt module\_op\_rmmod}
similarly free their third argument.
The CVS log message 
%%, nearly identical to those around it, 
simply reads ``fixed another memory leak.''





