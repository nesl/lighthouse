\section{A Need for Resource Management Validation}
\label{sec:mot}

Sensor network systems suffer from preventable mistakes caused by resources
mismanagement.
%
Both sensor network research and daily logs of system developers reveal the
span of this problem within the sensor network community.
%
This section reviews resource mismanagement from the perspective of those
battling it on a daily basis.



\subsection{Researchers Hardening Buffer Passing}


The TinyOS operating system is used to develop sensor network applications by
a diverse range of research and commercial organizations.
%
TinyOS applications often send data using the \code{SendMsg} interface.
%
This interface is used by calling the asynchronous \code{send} command to
send a \code{TOS\_Msg} to the network.
%
Rather than copying each message, the \code{SendMsg} interface passes a
reference to the \code{TOS\_Msg} to efficiently pass data through the protocol
stack.
%
Since TinyOS statically allocates resources, a \code{sendDone} event returns to
the caller a reference to the sent message to signal that the message send
completed.
%
It's important that the sending component not reuse the message buffer until
after the \code{sendDone} event, lest updates corrupt the inadvertently shared
buffer.



Research on interface contracts~\cite{archer07interface} recognizes this
management problem and proposes run time state monitoring as a solution.
%
Knowing that a message transferred via the \code{SendMsg} interface is either in
control of a user component or the TinyOS kernel, the \code{TOS\_Msg}
structure is augmented with a \code{msg\_state} field to track the message
as being either \code{USER\_OWNED} or \code{OS\_OWNED}.
%
A contract layer, interposed between the user component and the implementation
of \code{SendMsg}, verifies that components only \code{send} \code{USER\_OWNED} 
messages at run time.
%
The contract layer updates sent messages to the \code{OS\_OWNED} state until
the the \code{sendDone} event occurs when the message is returned to the
\code{USER\_OWNED} state.



This emergent research reveals academia's interest in improving resource
management safety for sensor network systems.



\subsection{Developers Struggling with Dynamic Memory}



SOS is a research operating system for sensor networks supporting dynamic
loading and unloading of software components, called modules, into deployed
systems.
%
The SOS kernel provides an API for these modules to dynamically allocate
memory using \code{ker\_malloc} and \code{ker\_free}, related to their C
namesakes respectively allocating and disposing of memory.  
%
As in C, programmers are responsible for manually managing their dynamic
memory.  



The SOS module loader proved to be a source of instability during the early
days of SOS.
%
The module loader is a core part of SOS, allowing users to dynamically add
modules into the running system.
%
Looking back through CVS logs reveals an interesting story of resource
mismanagement surrounding early development of the module loader.



\begin{figure}[tp]
\begin{scriptsize}
\begin{verbatim}
mod_op = (sos_module_op_t*) ker_msg_take_data(msg);
if(mod_op == NULL) return -ENOMEM;
if(mod_op->op == MODULE_OP_INSMOD) {
    existing_module = ker_get_module(mod_op->mod_id);
    if(existing_module != NULL) {
        uint8_t ver = sos_read_header_byte(
                existing_module->header,
                offsetof(mod_header_t, version));
            if (ver < mod_op->version) {
                ker_unload_module(existing_module->pid, 
                        sos_read_header_byte(
                        existing_module->header,
                        offsetof(mod_header_t, version)));
            } else {
                return SOS_OK;
            }
        }
    ret = fetcher_request(KER_DFT_LOADER_PID,
            mod_op->mod_id,
            mod_op->version,
            entohs(mod_op->size),
            msg->saddr);
    s->pend = mod_op;
    ker_led(LED_RED_TOGGLE);
    return SOS_OK;
}
return SOS_OK;
\end{verbatim}
\end{scriptsize}
\caption{\label{fig:leak}A memory leak in an SOS module.}
\end{figure}



In mid-October 2005 the block of code shown in figure~\ref{fig:leak} was
checked into CVS as part of \code{loader.c} and introduced a memory leak into
the loader.  
%
All paths through this block of code leak the \code{mod\_op} pointer, which
the module acquires exclusive ownership to via the \code{ker\_msg\_take\_data}
call.  



After three additional revisions not significantly modifying or fixing the
memory leak, a fourth revision was made in mid-December 2005.  
%
This revision expands the functionality of \code{loader.c} and breaks the code
up into smaller functions:



\begin{footnotesize}
\begin{verbatim}
sos_module_op_t *mod_op;
if (msg->saddr == ker_id() || s->pend) {
    return SOS_OK;
}
mod_op = (sos_module_op_t*) ker_msg_take_data(msg);
if(mod_op == NULL) return -ENOMEM;
switch(mod_op->op){
case MODULE_OP_INSMOD:
    return module_op_insmod(s,msg,mod_op);
case MODULE_OP_RMMOD:
    return module_op_rmmod(s,msg,mod_op);
}
return SOS_OK;
\end{verbatim}
\end{footnotesize}



Unfortunately, \code{mod\_op} is still leaked if \code{mod\_op->op} does not
find a matching case in the {\tt switch} statement.
%
Further, neither \code{module\_op\_insmod} nor \code{module\_op\_rmmod} free
\code{mod\_op}, passed to them as the third formal parameter.
%
Again, all paths through this code leak the buffer claimed as a result of
calling \code{ker\_msg\_take\_data}.
%
A day later, two months after the memory leaks were first introduced, these
memory leaks were found and fixed:



\begin{footnotesize}
\begin{verbatim}
sos_module_op_t *mod_op;
if (msg->saddr == ker_id() || s->pend) {
    return SOS_OK;
}
mod_op = (sos_module_op_t*) ker_msg_take_data(msg);
if(mod_op == NULL) return -ENOMEM;
switch(mod_op->op){
case MODULE_OP_INSMOD:
    return module_op_insmod(s,msg,mod_op);
case MODULE_OP_RMMOD:
    return module_op_rmmod(s,msg,mod_op);
}
ker_free(mod_op);
return SOS_OK;
\end{verbatim}
\end{footnotesize}



As shown above, a call to \code{ker\_free} added before the final
\code{return}, properly disposes of \code{mod\_op}.
%
The functions \code{module\_op\_insmod} and \code{module\_op\_rmmod} were
similarly updated to free their third argument.  
%
The CVS log message simply reads ``fixed another memory leak.''



Stories such as this of the SOS module loader reveal that sensor network
developers are struggling with correct resource usage.
%
Application of the analysis presented in this paper will alert developers at
compile time to the problem described above, helping to educate developers
about incorrect programing habits and relieving developers of tedious hunts
for bugs.



\subsection{Memory Ownership}
\label{subsec:owner}



Many resource management problems, including those described above, share an
underlying theme.  
%
Resources are shared dynamically across multiple software components to
support simple and efficient access to the resources.  
%
However, often only one component requires access to a given resource at any
point in time.  
%
In other words, proper resource management naturally obeys an {\em ownership}
discipline.  
%
In this style, each resource has a unique owning component (or {\em owner}),
who has the sole capability to access the resource.  
%
The owner also has the responsibility to eventually dispose of the resource or
explicitly {\em transfer} ownership to another component.



In this paper, we describe a simple, scalable, and practical static analysis
to enforce an exclusive ownership discipline on sensor network applications.
%
The analysis is made possible by a static ownership discipline resting upon
the following rules:

\begin{enumerate}
%
\item Each function may only refer to dynamic memory that it owns.  This
includes memory allocated in the function, memory in the encapsulating
component's persistent store, and any formal parameters known to release
dynamic memory into into the function body.
%
\item Each function that allocates or obtains ownership of dynamic memory must
eventually free that memory, transfer its ownership to another component, or
store the memory in a persistent location.
%
\item After a function frees or releases memory that it owns, it may no longer
access the memory.
%
\end{enumerate}



Our rules are necessarily an approximation of the run time requirements for
proper resource ownership, due to the modular nature of the rules and the
fundamental limitations of static analysis.  
%
\mynote{Roy: Tie above sentence into three primary goals?}
%
As described in the next section, the rules above can lead to both false
positives and false negatives.  
%
However, our experimental results indicate that these simple rules are a
practical approach for detecting resource management errors in real sensor
network software.

