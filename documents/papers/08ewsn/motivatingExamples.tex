\section{A Need for Resource Management Validation}
\label{sec:mot}

Sensor network systems suffer from preventable mistakes caused by resources
mismanagement.
%
Both sensor network research and daily logs of system developers reveal the
span of this problem within the sensor network community.
%
This section introduces two representative sources of resource mismanagement
from the currently used sensor network operating systems.



\subsection{Researchers Hardening Buffer Passing in TinyOS}
\label{ssec:tinyos}


The TinyOS operating system is used to develop sensor network applications by
a diverse range of research and commercial organizations.
%
TinyOS applications often send data using the \code{SendMsg} interface.
%
This interface is used by calling the asynchronous \code{send} command to
send a \code{TOS\_Msg} to the network.
%
Rather than copying each message, the \code{SendMsg} interface passes a
reference to the \code{TOS\_Msg} to efficiently pass data through the protocol
stack.
%
Since TinyOS statically allocates resources, a \code{sendDone} event returns to
the caller a reference to the sent message to signal that the message send
completed.
%
It's important that the sending component not reuse the message buffer until
after the \code{sendDone} event, lest updates corrupt the inadvertently shared
buffer.



Research on interface contracts~\cite{archer07interface} recognizes this
management problem and proposes run time state monitoring as a solution.
%
Knowing that a message transferred via the \code{SendMsg} interface is
either in control of a user component or the TinyOS kernel, the
\code{TOS\_Msg} structure is augmented with a \code{msg\_state} field to
track the message as being either \code{USER\_OWNED} or \code{OS\_OWNED}.
%
A contract layer, interposed between the user component and the
implementation of \code{SendMsg}, verifies that components only \code{send}
\code{USER\_OWNED} messages at run time.
%
The contract layer updates sent messages to the \code{OS\_OWNED} state until
the \code{sendDone} event occurs when the message is returned to the
\code{USER\_OWNED} state.
%
Incorrectly attempting to send a \code{OS\_OWNED} message results in the
interface contract generating a runtime fault.



This emergent research reveals academia's interest in improving resource
management safety for sensor network systems.



\subsection{Developers Struggling with Dynamic Memory in SOS}



SOS is a research operating system for sensor networks supporting dynamic
loading and unloading of software components, called modules, into deployed
systems.
%
The SOS kernel provides an API for these modules to dynamically allocate
memory using \code{ker\_malloc} and \code{ker\_free}, related to their C
namesakes respectively allocating and disposing of memory.  
%
As in C, programmers are responsible for manually managing their dynamic
memory.  



The SOS module loader proved to be a source of instability during the early
days of SOS.
%
The module loader is a core part of SOS, allowing users to dynamically add
modules into the running system.
%
The analysis described in this paper was applied to old CVS versions of the
module loader and quickly isolated bugs that had hindered initial development
of the loader.



\begin{figure}[tp]
\begin{scriptsize}
\begin{verbatim}
mod_op = (sos_module_op_t*) ker_msg_take_data(msg);
if(mod_op == NULL) return -ENOMEM;
if(mod_op->op == MODULE_OP_INSMOD) {
    existing_module = ker_get_module(mod_op->mod_id);
    if(existing_module != NULL) {
        uint8_t ver = sos_read_header_byte(
                existing_module->header,
                offsetof(mod_header_t, version));
            if (ver < mod_op->version) {
                ker_unload_module(existing_module->pid, 
                        sos_read_header_byte(
                        existing_module->header,
                        offsetof(mod_header_t, version)));
            } else {
                return SOS_OK;
            }
        }
    ret = fetcher_request(KER_DFT_LOADER_PID,
            mod_op->mod_id,
            mod_op->version,
            entohs(mod_op->size),
            msg->saddr);
    s->pend = mod_op;
    ker_led(LED_RED_TOGGLE);
    return SOS_OK;
}
return SOS_OK;
\end{verbatim}
\end{scriptsize}
\caption{\label{fig:leak}A memory leak in an SOS module.}
\end{figure}



In mid-October 2005 the block of code shown in figure~\ref{fig:leak} was
checked into CVS as part of \code{loader.c} and introduced a memory leak into
the loader.  
%
The SOS function \code{ker\_msg\_take\_data} transfers ownership a buffer,
passed to the function, from the kernel to the calling component.
%
From the users perspective, this is identical to allocating a new buffer and
copying into it the contents of an reference to external memory.
%
Application of our analysis generates warnings for all paths through this
block of code, which leaks the \code{mod\_op} pointer acquired via the
\code{ker\_msg\_take\_data} call.  
%
The bug slowly leaks memory storing module load and remove requests and, after
extended use of module loading and unloading, leads to node failure when no
memory is left available.



After three additional revisions to the module loader not significantly
modifying or fixing the memory leak, a fourth revision was made in
mid-December 2005.  
%
This revision expands the functionality of \code{loader.c} and breaks the code
up into smaller functions:



\begin{footnotesize}
\begin{verbatim}
sos_module_op_t *mod_op;
if (msg->saddr == ker_id() || s->pend) {
    return SOS_OK;
}
mod_op = (sos_module_op_t*) ker_msg_take_data(msg);
if(mod_op == NULL) return -ENOMEM;
switch(mod_op->op){
case MODULE_OP_INSMOD:
    return module_op_insmod(s,msg,mod_op);
case MODULE_OP_RMMOD:
    return module_op_rmmod(s,msg,mod_op);
}
return SOS_OK;
\end{verbatim}
\end{footnotesize}



Unfortunately, \code{mod\_op} is still leaked if \code{mod\_op->op} does not
find a matching case in the {\tt switch} statement.
%
Further, neither \code{module\_op\_insmod} nor \code{module\_op\_rmmod} free
\code{mod\_op}, passed to them as the third formal parameter.
%
Again, our analysis generates warnings stating that all paths through this
code leak the buffer claimed as a result of calling
\code{ker\_msg\_take\_data}.
%
A day later, two months after the memory leaks were first introduced, these
memory leaks were found and fixed:



\begin{footnotesize}
\begin{verbatim}
sos_module_op_t *mod_op;
if (msg->saddr == ker_id() || s->pend) {
    return SOS_OK;
}
mod_op = (sos_module_op_t*) ker_msg_take_data(msg);
if(mod_op == NULL) return -ENOMEM;
switch(mod_op->op){
case MODULE_OP_INSMOD:
    return module_op_insmod(s,msg,mod_op);
case MODULE_OP_RMMOD:
    return module_op_rmmod(s,msg,mod_op);
}
ker_free(mod_op);
return SOS_OK;
\end{verbatim}
\end{footnotesize}



As shown above, a call to \code{ker\_free} added before the final
\code{return}, properly disposes of \code{mod\_op}.
%
The functions \code{module\_op\_insmod} and \code{module\_op\_rmmod} were
similarly updated to free their third argument.  
%
The CVS log message simply reads ``fixed another memory leak.''
%
The analysis described in this paper would have prevented these eight weeks of
system instability by alerting developers when the initial memory leak was
introduced to the code base.



Stories such as this of the SOS module loader reveal that sensor network
developers are struggling with correct resource usage.
%
Application of the analysis presented in this paper will alert developers at
compile time to the problem described above, helping to educate developers
about incorrect programing habits and relieving developers of tedious hunts
for bugs.


