\section{A Need for Resource Management Validation}
\label{sec:mot}

Sensor network systems suffer from preventable mistakes caused by resources
mismanagement.
%
Both sensor network research and daily logs of system developers reveal the
span of this problem within the sensor network community.
%
This section introduces two representative sources of resource mismanagement
from currently used sensor network operating systems.



\subsection{Buffer Passing in TinyOS}
\label{ssec:tinyos}


The TinyOS operating system is used to develop sensor network applications by
a diverse range of research and commercial organizations.
%
TinyOS applications often send data using the \code{SendMsg} interface.
%
Interface users call the asynchronous \code{send} command to send a
\code{TOS\_Msg} to the network.
%
Rather than copying each message, the \code{SendMsg} interface passes a
reference to the \code{TOS\_Msg} to efficiently pass data through the protocol
stack.
%
Since TinyOS statically allocates resources, a \code{sendDone} event returns
to the caller a reference to the sent message to signal that the message send
completed.
%
The sending component should not reuse the message buffer until after the
\code{sendDone} event, lest buffer writes corrupt the inadvertently shared
buffer.



Research on interface contracts~\cite{archer07interface} proposes run time
state monitoring as a solution to the management problem.
%
Knowing that a message transferred via the \code{SendMsg} interface is
either in control of a user component or the TinyOS kernel, the
\code{TOS\_Msg} structure is augmented with a \code{msg\_state} field
tracking the message as either \code{USER\_OWNED} or \code{OS\_OWNED}.
%
A contract layer, interposed between the user component and the
\code{SendMsg} implementation, verifies that components only \code{send}
\code{USER\_OWNED} messages at run time.
%
The contract layer updates sent messages to the \code{OS\_OWNED} state until
the \code{sendDone} event occurs, when the message is returned to the
\code{USER\_OWNED} state.
%
Incorrect attempts to send a \code{OS\_OWNED} message results in the
interface contract generating a run time fault.



This emergent research reveals academia's interest in improving resource
management safety for sensor network systems.  
%
Our tool statically alerts developers to this type of resource
mismanagement.



\subsection{Dynamic Memory in SOS}



SOS is a research operating system for sensor networks supporting dynamic
loading and unloading of software components, called modules, into deployed
systems.
%
The SOS kernel provides an API to dynamically allocate memory using
\code{ker\_malloc} and \code{ker\_free}, respectively allocating and
free memory.  
%
As in C, programmers are responsible for manually managing their dynamic
memory.  


The following example demonstrates a representative example of memory
mismanagement from the SOS operating system.
%
The module loader is a core part of SOS, allowing users to dynamically add
modules into the running system.
%
Unfortunately, the SOS module loader proved to be a source of instability
during the early days of SOS.
%
The tool described in this paper was applied to old CVS versions of the
module loader and quickly isolated bugs that had hindered initial
development of the loader.



\begin{figure}[tp]
\centering
\lstset{numbers=left, language=C}
\begin{lstlisting}
mod_op = (sos_module_op_t*) ker_msg_take_data(msg);
if(mod_op == NULL) return -ENOMEM;
if(mod_op->op == MODULE_OP_INSMOD) {
    existing_module = ker_get_module(mod_op->mod_id);
    if(existing_module != NULL) {
        uint8_t ver = sos_read_header_byte( ... );
            if (ver < mod_op->version) {
                ker_unload_module(  ... );
            } else {
                return SOS_OK;
            }
        }
    ret = fetcher_request( ... );
    s->pend = mod_op;
    ker_led(LED_RED_TOGGLE);
    return SOS_OK;
}
return SOS_OK;
\end{lstlisting}
\caption{\label{fig:leak}A memory leak in an SOS module.}
\end{figure}



In mid-October 2005 the block of code shown in figure~\ref{fig:leak} was
checked into CVS as part of \code{loader.c} and introduced a memory leak into
the loader.  
%
The SOS function \code{ker\_msg\_take\_data} transfers ownership of a buffer,
passed to the function, from the kernel to the calling component.
%
From the user's perspective, this is identical to allocating a new buffer.
%
Application of our analysis generates warnings for all paths through this
block of code, which leaks the \code{mod\_op} pointer acquired via the
\code{ker\_msg\_take\_data} call.  
%
The bug slowly leaks memory storing module load and remove requests and, after
extended use of module loading and unloading, leads to node failure when no
memory is left available.



After three additional revisions to the module loader not significantly
modifying or fixing the memory leak, a fourth revision was made in
mid-December 2005.  
%
This revision expanded the functionality of \code{loader.c} and broke the
code in figure~\ref{fig:leak} into individual functions to handle module
insertion requests and module removal requests.
%
Unfortunately, as our analysis found, \code{mod\_op} continued to be leaked
on all paths through the updated code.
%
A day later, two months after the memory leaks were first introduced, these
memory leaks were found and fixed by adding calls to \code{ker\_free} on
each path through the function.
%
The CVS log message simply reads, ``Fixed another memory leak.''
%
Lighthouse would have prevented these eight weeks of system instability by
alerting developers when the initial memory leak was introduced to the code
base.



Experiences, such as this of the SOS module loader, reveal that sensor network
developers are struggling with correct resource usage.
%
Application of the tool presented in this paper will alert developers at
compile time to the problem described above, helping to educate developers
about incorrect programming habits and relieving developers of tedious hunts
for bugs.


