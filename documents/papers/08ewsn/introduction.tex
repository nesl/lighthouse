\section{Introduction}
\label{sec:intro}

% Update to include three motivating design goals for our work:
% 
% - Simplicity
% - Scalability
% - Practicality
% 
% Simplicity appears in the exclusive ownership model.  The model is easy
% for programmers to understand and work with.  This simple model provides
% very good information about programs (as seen in the evaluation) without
% excessive analysis complexity.
% 
% Scalability motivates our use of function specifications that describe
% pre- and post- conditions.  Our analysis simply examines one function at a
% time.
% 
% Practicality is demonstrated through the evaluation that locates a number
% of bugs in SOS with a low level of false positives, despite the simple
% nature of the analysis.


Networked embedded systems, or sensor networks, are finding ubiquitous
application in densely sampling phenomena --- from structural properties of
buildings to wildlife behavior --- that were previously difficult or
impossible to observe.  
%
Like embedded systems, sensor networks operate in resource and energy
constrained environments with minimal operating system support.  
%
However, unlike traditional ``dedicated'' embedded applications, sensor
network applications are reprogrammable, and applications may change in the
field.  
%
For this reason, sensor network applications can benefit from operating
system support for general-purpose system abstractions.



A useful class of system abstractions provides facilities to dynamically
manipulate system resources.  
%
Such abstractions simplify application development and allow an application
to naturally and efficiently respond to the changing needs of its
environment.  
%
For example, TinyOS~\cite{TinyOS} uses a buffer-swapping protocol between
components for efficient sharing of statically allocated buffers.  
%
The SOS operating system~\cite{sos} supports dynamic memory allocation,
while MANTIS OS~\cite{abrach03mantis} supports both dynamic memory
allocation and thread creation.  
%
Frameworks built on top of sensor network systems also include components
that make resources dynamically available to other parts of the system.  
%
An example of this form of dynamic resource management is the
VanGo~\cite{greenstein05vango} framework's buffer pool for TinyOS. 



While the ability to manipulate resources increases the expressiveness of
sensor network applications, this expressiveness can be a double-edged
sword.  
%
Improper management of resources leads to subtle errors affecting both the
correctness and efficiency of applications.
%
For example, consider the architecture of a simple sensor network sense and
send application.  
%
Like many sensor network applications, this application can be arranged in
a dataflow architecture:  raw sensed data captured at the sensors moves
through various filters before being sent into the network.  
%
Data transfer through filters is naturally and efficiently implemented by
passing references between system components.



Unfortunately, incorrect implementations of buffer passing lead to serious
errors.  
%
First, a component may access a (dangling) reference to data that has been
passed downstream and possibly freed.
%
This is particularly dangerous on the embedded processors without memory
management units, where a bad dereference can subtly corrupt data or crash
the system. 
%
Second, a component may receive a data buffer from an upstream component but
forget to either dispose it or pass it on to the next stage for processing.
%
Since expensive garbage-collection mechanisms are not available, such
resource leaks will rapidly exhaust available memory.



Along with others in the sensor network community~\cite{07interface,XXX}, we
observe that a clear cut exclusive ownership discipline fits many aspects of
sensor network systems.
%
In this style, every resource has exactly one {\em owner} component at any
point.  
%
The component creating the resource (dynamically through a kernel call or
statically through a variable declaration) assumes initial ownership.  
%
Each component has the right to access the resources that it owns but also
the responsibility to eventually either free the resource or transfer
ownership to another component.  
%
A component may not manipulate resources that it does not own.  
%
When all components obey this protocol, then there can be neither accesses
to dangling resources nor resource leaks.
%
Current sensor network systems do not provide a mechanism for helping
programmers to ensure that this ownership protocol is properly obeyed.  
%
Our experience has been that programmers often make subtle mistakes in
ownership, leading to critical and difficult-to-track errors.  



In this paper, we present an approach for automated validation of dynamic
resource management in sensor network software.  
%
Our design rests upon four primary goals:
%
\begin{description}
%
\item[Simplicity]  The analysis uses a simple exclusive ownership model.
%
This allows developers to understand the bugs found by the analysis, rather
than simply fix them.
%
\item[Statically] The analysis is performed statically at compile time.  This
provides developers feedback early in their design cycle and prevents bugs
from entering into deployed networks.
%
\item[Modularity]  A simple pre- and post- condition specification
facilitates module analysis.
%
\item[Practicality] The analysis finds numerous bugs in real sensor network
systems.
%
\end{description}
%



We have instantiated these ideas in Lighthouse, a tool for statically
ensuring proper management of dynamic memory in the SOS operating
system~\cite{sos}.  
%
Lighthouse uses a dataflow analysis to analyze each software component {\em
at compile time} for violations of the ownership protocol, providing early
feedback to programmers about the potential for dynamic resource errors.  
%
The dataflow analysis ensures key ownership invariants:  each resource has a
unique owner; each resource is only manipulated by its owner; and each
resource is either freed by its owner or transferred to another owner.  
%
Modular analysis is achieved via an external specification documenting
system memory manipulation functions that summarizes the effect of function
calls.
%
While rarely used or required, end programmers can provide additional
application specific information to refine the analysis results.
%
With our approach, sensor network programmers use the expressiveness of
dynamic resource management while retaining confidence in the reliability of
their applications.



The exclusive ownership discipline is easy to understand and made explicit to
developers.
%
Rather than acting as a black box, the approach presented in this paper
provides a clear framework for sensor network developers to reason about
their systems.


We focus on SOS memory management because of our experience with memory
errors as users of SOS, which was our initial motivation.  
%
However, we believe the underlying dataflow analysis is more broadly
applicable, and Lighthouse is parametrized by the API for resource creation,
ownership transfer, and resource deletion.  
%
Therefore, we believe that our tool can be easily adapted both to statically
track ownership for other kinds of resources and to handle C-based operating
systems other than SOS.



We evaluated Lighthouse on the historic versions of all user modules in the
SOS CVS repository, as well as on the SOS kernel.  
%
Overall, we ran the checker on about 40,000 lines of code.  
%
The tool identified 25 suspect memory operations of which eight were actual
memory errors in the historic versions of the user modules.
%
This analysis used a shared specification documenting 9 functions
manipulating SOS, and 9 additional application specific specifications
describing user functions that manipulate memory.
%
Within the SOS kernel our analysis located 35 memory operations of which two
were actual memory errors in the SOS kernel, using a specification
documenting an additional 17 internal operating system functions.  
%
The latter finding is surprising, since the SOS kernel code is
relatively mature and was written by experienced systems programmers.  
%
In summary, our experiments illustrate the practicality of our approach for
providing early feedback about memory errors in real sensor network
software.



The rest of the paper is structured as follows.  
%
Section~\ref{sec:mot} further motivates the need for this analysis by
examining other recent sensor network research and citing examples from
system development.
%
Section~\ref{sec:alg} presents a precise description of the exclusive
ownership protocol and how Lighthouse enforces it.  
%
Section~\ref{sec:eval} provides our experimental results, which illustrate
the utility of our tool in practice.  
%
Section~\ref{sec:related} compares against related work, and
Section~\ref{sec:conc} concludes.



