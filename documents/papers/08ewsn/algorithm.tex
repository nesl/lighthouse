\section{The Lighthouse Tool}
\label{sec:alg}



A discipline of {\em exclusive data ownership} underlies the examples
provided in section~\ref{sec:mot}.
%
References to resources are passed between multiple components to support
simple and efficient resource access.
%
However, only one component accesses a given resource at any point in time.
%
Each resource in this discipline has a unique owning component with the sole
capability to access the resource.
%
The owner also has the responsibility to eventually dispose of the resource or
explicitly transfer ownership to another component.



In this section we describe our approach to statically enforcing an exclusive
ownership discipline on sensor network applications.  
%
Informally, the following rules guarantee conformance to this discipline:
%
\begin{enumerate}
%
\item Each function may only refer to dynamic memory that it owns.  This
includes memory allocated in the function, memory accessible to the function
via global variables, and any formal parameters transferring dynamic data ownership to
the component.
%
\item Each function allocating or obtaining ownership of dynamic memory must
eventually free that memory, transfer its ownership to another component, or
store the memory in a persistent location.
%
\item After freeing or releasing memory, a function may no longer access the
memory.
%
\end{enumerate}



\subsection{TinyOS and SOS, Revisited}



The \code{SendMsg} interface described in section~\ref{ssec:tinyos} clearly
expects conformance to the above three rules.
%
Each component initially owns any \code{TOS\_Msg} that it declares.
%
Calls to the \code{send} command transfer ownership of the \code{TOS\_Msg}
from the component to the kernel, after which the component must not access
the message until ownership is regained via the \code{sendDone} event.
%
Interface contracts provide a means to verify these properties at {\em
runtime}.
%
Our work describes a simple technique to {\em statically} examine these same
properties.
 


The SOS kernel already dynamically tracks memory ownership and has an
associated API for ownership transfer, which inspired our work.  
%
Tracked ownership enables the SOS kernel to perform basic garbage collection
when a module is removed from a node.
%
Calls to the messaging API can specify the \code{SOS\_MSG\_RELEASE} flag to
signal that ownership of the posted message is to be transferred to the callee
(or freed by the kernel if the callee does not explicitly take ownership).  
%
Calls to \code{ker\_malloc}, \code{ker\_free}, and \code{ker\_msg\_take\_data}
are known to allocate, free, and transfer ownership of memory respectively.
%
The SOS kernel only uses data ownership information when a module is removed
from a node at run time and assumes that the API manipulating memory ownership
is properly used.
%
Our work makes this protocol explicit and provides static checking for
conformance.



\subsection{Ownership Specification}


%  Use specification file
%  - documentation
%  - concise
%  - simlpe
%  - reusable
%  
%  Specification describes pre- and post- memory conditions for function
%  - full / empty
%  - formal / return / store
%  - Examples for malloc / free / post 
%  
%  Global store



Analyzing memory ownership requires understanding how functions manipulate
memory ownership properties.
%
Our analysis uses a simple specification file documenting functions that
transfer memory ownership between components.
%
Each function body is checked to ensure safety under the assumption provided
by the specification.  



A function specification describes preconditions and postconditions relating
to the dynamic memory referenced by expressions.
%
These preconditions and postconditions describe expressions releasing
memory buffers to or from the function.
%
Expressions in a specification may be one or more of: 
%
\begin{itemize}
%
\item Formal variable identified by its index in the function prototype
%
\item Function return value identified by the keyword \code{return} (only
valid in postconditions)
%
\item Global variable identified by variable name
%
\end{itemize}
%
An expression annotated with \code{release} in the postcondition describes dynamic
memory that the caller of the function takes ownership of from the callee, and
therefore that the function body must release.
%
An expression annotated with \code{release} in the precondition describes dynamic
memory that the caller of the function releases to the callee, and therefore
that the function body must take ownership of.



\begin{figure}[tp]
\centering
\lstset{numbers=none, language=C}
\begin{lstlisting}
                ker_malloc.pre { $return.release; }

                ker_free.post { $1.release; }
\end{lstlisting}
\caption{\label{fig:spec}Specifications for \code{ker\_free} and
\code{ker\_malloc}}.
\end{figure}



Example specifications demonstrating these ideas are presented in
Figure~\ref{fig:spec}.
%
The postcondition qualifying the return statement with \code{release} notes
that the caller of \code{ker\_malloc} must take ownership of the returned
data.
%
The precondition qualifying the first formal parameter to \code{ker\_free}
specifies that it points to dynamic memory released from the caller to the function body.



The specification resides in a single external configuration file.
%
Only functions requiring annotations need be included in the specifications.
%
In practice we have found that a small set of annotations, \numannote for
the complete evaluation of SOS, is sufficient for precise analysis. 



The specification allows modular checking by describing the side effects of
called functions.
%
The ability to perform modular checking allows application writers to obtain
early feedback about the correctness of their resource management, without
requiring access to the rest of the system.  
%
This is particularly important in a system like SOS, in which modules can be
linked and unlinked dynamically.  
%
In such a setting, the ``rest'' of the system is a moving target, so it is not
really possible to consider an approach based on whole-program analysis.



\subsection{Implementation Overview}



Lighthouse is implemented in the CIL front end for C~\cite{CIL}, which
parses C code into a simple intermediate format and provides a framework for
performing analysis on the intermediate code. 
%
Lighthouse takes as input a preprocessed C file and prints out warning
messages similar to those produced by a C compiler when suspect code is
identified.
%
The analysis does not modify the preprocessed code, so it can be trivially
called from a makefile between the preprocessing and code generation stages
of compilation.



Lighthouse uses a dataflow analysis to track the state of all expressions that
may alias dynamic memory within a function.
%
The dataflow ensures that:
%
\begin{itemize}
%
\item Whenever a node in the control flow graph is encountered that allocates
or takes ownership of a block of memory, every path from this node to the
function exit frees, stores, or releases ownership of the memory exactly once.  
%
If this property is not satisfied, Lighthouse reports a possible memory leak.
%
\item Whenever a node in the graph is encountered that frees or releases
ownership of a block of memory, that no path from this node to the function
exit accesses the memory.  
%
If this property is not satisfied, Lighthouse reports a possible dangling
pointer error.
%
\end{itemize}
%
For the purpose of the Lighthouse traversal over a function's CFG, the
function's entry node is considered an allocation point for all
parameters qualified as with \code{release} in the preconditon, and the function's exit
nodes are considered release points for expressions qualified with
\code{release} in the postcondition.



The analysis described above requires knowledge of the memory pointed to by
a function's pointers.  
%
This is statically approximated by an alias analysis, which determines
whether two different pointers store the same memory location at a given
program point.  
%  %
%  Two standard approximations to the true dynamic alias information are {\em
%  must-alias} analysis and {\em may-alias} analysis.
%  %
%  We have built a simple flow-sensitive must-alias analysis for use by
%  Lighthouse.  
%  %
%  For the may-alias analysis, we use a fast flow-insensitive analysis provided
%  by the CIL framework.  
%  %
%  Obtaining precise alias information at compile time is notoriously
%  difficult, and this limitation is the principal cause of false positives for
%  our analysis.
%  %
%  For example, CIL's may-alias analysis does not distinguish among the fields
%  of a structure, instead considering them to always potentially alias one
%  another.  
%  %
%  Both alias analyses can be imprecise in the presence of linked data
%  structures.



\subsection{Limitations}



As we demonstrate in the next section, our checker is useful for detecting
violations of the ownership protocol on real sensor network code.  
%
However, the checker is not guaranteed to find all such violations.  
%
By favoring simplicity, scalability, and practicality, the checker allows some
false negatives.




First, we favored using a simple memory model at the cost of not precisely
handling all of the unsafe features of the C programming language.  
%
For example, pointer arithmetic is not statically analyzed.  
%
Instead, an expression of the form $p+i$, where $p$ is a pointer and $i$ is an
integer, is simply treated as if it refers to the same block of memory as $p$.  
%
If $p+i$ in fact overflows to another block of memory at run time, the
checker's assumption can cause it to miss errors.  
%
These kinds of memory safety assumptions are standard for C-based program
analysis.



Second, there is a design choice about how to treat dynamic memory referenced
by formal parameters without the \code{release} denotation.
%
Technically the ownership protocol disallows this memory from being accessed
at all, since the receiving component is not the owner.  
%
Lighthouse assumes a more practical stance by not enforcing the requirement
that a component only access a resource that it owns.
% 
This decision provides a mechanism for allowing patterns of resource sharing
that are in fact safe but not supported by the ownership discipline, for
example when a component temporarily ``borrows'' data from another component
within a bounded scope.
%
The cost of this flexibility is the potential to miss real resource management
errors.
%
However, the checker still ensures that a function does not access a resource
that has earlier been released by the function, thereby detecting memory
leaks.



Third, sensor network systems are often event driven with unpredictable event
orderings.
%
Our modular checker makes optimistic assumptions about the order in which
events will be received by a component.  
%
However, event orderings can affect the correctness of ownership tracking when
a resource is accessed via the persistent store.
%
For example, consider a component that responds to three events: $\create$
causes the component to allocate a block $b$ and store it into the component's
persistent store, $\access$ causes the component to access block $b$ via the
store, and $\delete$ causes the component to deallocate block $b$ from the
store.  
%
The component avoids dangling accesses to $b$ and leaks of $b$ as long as the
temporal sequence of events follows the regular expression $($ $\alloc$
$\access^*$ $\delete$ $)^*$.  



Preferring a modular analysis that only requires examining a single function
at a time, Lighthouse only tracks data locally within each event handler.  
%
When performing this tracking, the tool assumes on entry that all data in the
persistent store is owned by the component.
%
The tool similarly considers a resource to be properly released when it is
placed in the persistent store.  
%
These assumptions allow each of the three event handlers in our example to
pass all checks.  
%
Nonetheless, dynamic memory errors can still happen, for example if an
$\access$ event ever occurs before the $\alloc$ event.  
%
Our assumptions provide a practical point in the design space that allows each
event handler to be usefully checked for local violations of the ownership
protocol.  



We have explored and implemented a prototype system combining Lighthouse with
a finite state machine (FSM) representation of the expected event orderings.
%
This allows tracking of persistent store state across event handlers, solving
the problem described above.
%
However, this requires pushing basic monitoring into the system runtime to
watch for invalid event firing orders.
%
In practice, we found very few instances where the additional tracking of event
orderings helped refine our analysis of memory management in SOS.
%
More complete handling of inter-component relationships and application-level
protocols is something we plan to pursue in the future, leveraging recent work
on {\em interface synthesis} for software components~\cite{AlurPOPL05,HJM05}.

