\section{Conclusion}
\label{sec:conc}

Systems programming APIs for networked embedded systems applications
are getting more expressive as the applications become more
sophisticated.  This necessitates a programming environment that
supports program development by providing automated compile-time
checkers for proper API usage.  Support for such validation is
particularly important in the sensor network domain.  First,
sensor network software is intended to run under resource constraints
and unpredictable environments, making them more prone to error.
Second, the cost of fixing a bug in the field is
high.  Third, networked embedded systems come with somewhat limited
debugging support, making it extremely difficult to reason about the
cause for a failure in the field.

This paper is our first step toward providing static analysis tools
that capture common programming idioms in networked embedded systems.
We have focused on dynamic resource management initially, since our
experience with SOS suggests that improper memory management is a
common source of hard-to-debug problems.  We have demonstrated how our
approach can effectively find memory errors even in code written by
systems experts.  Lighthouse is now provided as part of the
SOS development environment; it is invoked automatically in the build
process.

In the future, we would like to augment the techniques in this paper
with support for reasoning about application-level protocols
\cite{AlurPOPL05,HJM05} as well as concurrency issues.  Ultimately,
our goal is to provide a systems programming environment where many
common classes of bugs are automatically detected at compile time.  We
believe such early feedback will help increase the productivity of
sensor network programmers and the reliability of their software.


