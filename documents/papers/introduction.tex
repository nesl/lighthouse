\section{Introduction}
\label{sec:intro}

Networked embedded systems, or sensor networks, are finding ubiquitous application in densely
sampling phenomena --- from structural properties of buildings to 
wildlife behavior --- that were previously difficult or impossible to observe.
Like embedded systems, sensor networks operate in resource and energy 
constrained environments with minimal operating system support.
However, unlike traditional ``dedicated'' embedded applications, sensor network
applications are reprogrammable, and applications may change in the field. 
For this reason, sensor network applications can benefit from
operating system support for 
general-purpose system abstractions.

A useful class of system abstractions provides facilities
to dynamically create, manage, and destroy system resources.
Such abstractions can simplify application
development and allow an application to naturally respond to
the changing needs of its environment.  Several platforms for sensor networks
provide a form of dynamic resource management.  For
example, the SOS operating system~\cite{sos} supports dynamic
allocation of memory, while MANTIS OS~\cite{abrach03mantis} supports both dynamic
memory allocation and thread creation.  Frameworks built on top of sensor
network systems also include components
that make resources dynamically available to other parts of
the system.  The buffer pool
used in the VanGo framework~\cite{greenstein05vango} for TinyOS~\cite{TinyOS}
is an example of this form of dynamic resource management.

While the ability to dynamically manipulate resources increases the
expressiveness of sensor network applications, this expressiveness can
be a double-edged sword.  Improper management of resources can lead to
subtle errors that can affect both the correctness and efficiency of
applications.


For example, consider the architecture of a simple
sensor-network application,
shown in Figure~\ref{fig:surge-dataflow}.
Like many sensor-network applications, this application
is arranged in a dataflow architecture:  raw sensed
data captured at the sensors moves through various filters (via the
{\tt surge} module) before being forwarded to a base station (via the
{\tt tree routing} module).  
In order to naturally and
efficiently implement the message passing, data buffers
are dynamically allocated as necessary and
passed by reference rather than by copying.

If incorrectly implemented, this style of data sharing can lead to serious errors.
First, a module may access a (dangling) reference to data
that has been passed on and possibly freed downstream. 
%
The low-end processor hardware used in current sensor nodes does not
have a memory management unit.  Thus existing sensor node operating systems 
for these platforms,
like TinyOS~\cite{TinyOS} and SOS~\cite{sos}, do not support
virtual memory, and a bad dereference can crash the system. 
%
Second, a module may get a data buffer from an upstream module
but forget to either
release it or to pass it on to the next stage for processing.
Since expensive garbage-collection mechanisms are not available, such resource
leaks will very soon exhaust available memory.

A key insight in these kinds of systems is that data transfer between modules
typically does not lead to sharing, but instead follows a
producer-consumer pattern.  Therefore, sensor network
programmers typically implement correct and efficient resource
management using implicit {\em ownership-based} data access protocols.  In
this style, every resource has exactly one {\em
owner} module at any point.  The module that creates the resource
(through some kernel call) assumes initial ownership.  Ownership
of a resource is explicitly transferred through kernel calls that
implement message passing.
%% Dually, a module that receives a message
%% can explicitly acquire ownership of the data.
Each module has the
right to access the resources that it owns, but also the
responsibility to either free the resource or transfer ownership to
another module.  A module may not manipulate resources that it does
not own.  If all modules obey this protocol, then there can be neither
accesses to dangling pointers nor resource leaks.
%% Further, once ownership is transferred, the module cannot access the
%% resource, e.g., by dereferencing the pointer to the resource.
%% Together, the protocol guarantees that there are no accesses of
%% dangling pointers, nor are there resource leaks.

While especially important for the proper management of dynamic
resources, this kind of ownership model is also useful for managing
static resources.  
For example, TinyOS lacks support for
dynamic memory allocation, but statically allocated buffers are often
transferred among components for efficiency reasons.
TinyOS employs a split-phase approach to message
passing~\cite{TinyOS}:
a message
pointer is passed via a call to {\tt Send.send}, and the sender
receives the {\tt Send.sendDone} event when message transmission has
completed.
The calling component should not
modify the message pointer in any way until receiving the {\tt sendDone}
event.  Ownership is a natural protocol for guaranteeing this
behavior:  the calling component transfers ownership of the pointer
to the callee upon a {\tt
  send} call and regains ownership upon the {\tt sendDone} event~\cite{ownershipthread}.


Current sensor-network systems do not help
programmers to ensure that ownership protocols are properly obeyed.
In most systems, these protocols are completely implicit and can
be expressed only informally through programming conventions and comments.
The SOS operating system includes an explicit API for expressing
ownership relationships.  However, programmers must take
responsibility for ensuring that this API is properly used.
Our experience has been that programmers often make
mistakes in implementing ownership protocols, leading to critical
and difficult-to-track errors.  %% These problems are exacerbated in the
%% context of sensor networks, where the target customers (i.e., the
%% application writers) are scientists who may lack systems
%% programming expertise.

In this paper, we present a tool for automated validation of dynamic
resource management in sensor-network software.
Programmers employ an explicit API along with
lightweight program annotations to express ownership intentions for
resources.
Our tool checks each module {\em at compile time} for violations of
the ownership protocol, providing early feedback
to programmers about the potential for dynamic memory errors.  The
tool employs a novel suite of dataflow analyses to check the key ownership
invariants:  each resource has a unique owner; each resource is only
manipulated by its owner; and each resource is either freed by its
owner or transferred to another owner.  %% Together these invariants
%% ensure proper management of resources at run time.
With our tool, sensor-network programmers can make use of the
expressiveness of dynamic resource management 
while retaining confidence in the reliability of their applications.

We have instantiated our approach as a tool for statically ensuring
proper management of dynamic memory in the SOS operating
system~\cite{sos}, which employs the C programming language.
We focus on SOS memory management initially because of our experience
with memory errors as users of SOS, which was our primary motivation.  
However, the underlying dataflow analyses are generic, and
our tool is parametrized
by the API for resource creation, ownership transfer, and resource deletion.
Therefore, we believe that our tool can be easily adapted
both to statically track ownership for other kinds of resources and to
handle C-based operating systems other than SOS.

We evaluated our tool on the historic versions of all
user modules in the SOS CVS repository, as well as on the SOS kernel.
Overall, we ran the checker on about 46,000 lines of code.
%% Analysis of the historic modules reveals that the
%% tool is able to locate improper memory management in real
%% sensor-network software.
The tool identified
88 suspect memory operations of which 16 were actual
memory errors in the historic
versions of the user modules, as well as 28 suspect memory operations of which 
two were actual memory errors in the SOS kernel.  
The latter finding is
somewhat surprising, since the SOS kernel code is relatively mature
and was written by systems programming experts.
In summary, our experiments illustrate the practicality of our
approach for providing early feedback about memory errors in
real sensor-network software.

The rest of the paper is structured as follows.
Section~\ref{sec:example} illustrates the problems for dynamic
resource management via a small example and informally describes our ownership
protocol that resolves these problems.  Section~\ref{sec:alg} presents
a precise description of this protocol and the way in which our tool
enforces it.  Section~\ref{sec:eval} provides our experimental
results, which illustrate the utility of our tool in practice.
Section~\ref{sec:related} compares against related work, and
Section~\ref{sec:conc} concludes.

%% \smallskip
%% \noindent
%% {\bf Limitations.}
%% %
%% There are two main limitations of our tool.  First,
%% the precision of the tool depends on the precision of an underlying
%% {\em alias analysis}, which is used to statically
%% reason about the relationships
%% among pointers.
%% Our tool currently uses an
%% off-the-shelf alias analysis provided by CIL~\cite{CIL}, a front end
%% for C.  While alias analysis can be notoriously imprecise in the
%% presence of complex pointer manipulations and pointer structures, 
%% our experimental results indicate that
%% aliasing is not a major source of false positives in this class of 
%% programs.  This positive result is likely due to
%% the stylized ways in which message pointers are
%% manipulated in sensor-network applications.  

%% Second, %% our tool currently only tracks ownership properties 
%% %% on each event handler in isolation, without considering the overall
%% %% order in which events will be received.
%% %% However,
%% there may be application-level protocols that determine how events
%% are generated in the system, and these protocols can affect the
%% correctness of ownership tracking.
%% %
%% For example, imagine a module that responds to three events:  $\alloc$
%% causes the module to allocate a block $b$ and store
%% it into the module's persistent store,
%% $\access$ causes the module to access block $b$ via the  store,
%% and $\delete$ causes the module to deallocate block $b$ via the 
%% store.
%% The module properly manages block $b$ as long as the temporal
%% sequence of events follows the regular expression $\alloc$ $\access^*$
%% $\delete$.  This ordering ensures that the system never accesses the
%% block before it is allocated and always eventually deletes the block.

%% Our tool currently does not reason across event handlers, instead only
%% tracking data locally within each event handler.
%% When tracking ownership within a handler, the tool assumes on entry
%% that all data in the persistent store is owned by the module.  This
%% assumption can miss errors, for example if an $\access$ event ever
%% occurs before the $\alloc$ event.  However, it is a practical
%% design choice that allows each event handler to be usefully checked
%% for local violations of the ownership protocol.
%% %
%% It will be interesting to extend our system with additional interface
%% annotations about application-level protocols in the future~\cite{AlurPOPL05}.
